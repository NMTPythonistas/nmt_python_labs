% LAB 3: Functions
%
% CSE/IT 107: Introduction to Programming
% New Mexico Tech
%
% Prepared by Russell White and Christopher Koch
% Spring 2015

% - Functions
%   - Positional and Optional Arguments
%   - Recursion
% - Modules
%   - import statements
%   - main() boilerplate
% - Comments and PEP-8 style
% - Keywords: def
\documentclass[11pt]{cselabheader}
\fancyhead[R]{Lab 3: Functions}
\title{Lab 3: Functions}
\usepgflibrary{fpu}
\newcommand{\pas}{\mathop{\mathrm{pascal}}}

\begin{document}
\pagenumbering{roman}
\maketitle

\hrule
\begin{quotation}
``If you don't think carefully, you might believe that programming is just
typing statements in a programming language.''
\end{quotation}
\begin{flushright}
  --- W. Cunningham
\end{flushright}

\begin{quotation}
``When you come to a fork in the road, take it."
\end{quotation}
\begin{flushright}
--- Attributed to Yogi Berra
\end{flushright}

\begin{quotation}
``The time you enjoy wasting is not wasted time.''
\end{quotation}
\begin{flushright}
  --- Bertrand Russell
\end{flushright}
\hrule

\tableofcontents

\pagebreak
\pagenumbering{arabic}

\section{Introduction}
\label{sec:intro}

In the previous lab, we showed you simple control flow and how to repeat a piece
of code using \pythoninline{while}. In this lab, we will be learning how to
break a lot of code into smaller, reusable pieces called \emph{functions}.



\pagebreak
\section{Recursion}
\label{sec:rec}

Recursion is the idea of a function calling itself. This can be useful in a wide
variety of situations, but it can also be easy to misuse. Let us first look at a
good example:
\begin{python3code}
def factorial(n):
    """Compute and return n! recursively by knowing that n! = n * (n-1)! for n > 1
    and 1! = 1 and 0! = 1"""
    if n <= 1:
        return 1
    else
        return n * factorial(n-1)
\end{python3code}

Notice that this is just another way to repeat a statement, similar to a loop.
In a loop, we have to specify when to stop repeating something, and we have to
do the same thing with recursion; otherwise, it would run forever.

As an example, this is how the function plays out:
\[ f(5) = 5 * f(4) = 5 * 4 * f(3) = 5 * 4 * 3 * f(2) = 5 * 4 * 3 * 2 * f(1) = 5
* 4 * 3 * 2 * 1 = 120 \]

As an example, see this simple \textbf{\emph{bad}} case of recursion:

\begin{python3code}
def test():
    """BAD example of recursion."""
    test()
\end{python3code}

If this function is called, it will continue to call itself infinitely until
Python prints out an error message and quits. This is because the function calls
itself every time it is called -- there is no case where the function ends
without calling itself again. The case in which the recursion ends is called a
\emph{base case} and it is important to include one in every recursive function.
Here is a modified version of the above code that allows the initial call to
specify a value limiting how many times the recursive call should be made:

\begin{python3code}
def test(x):
    if x > 0:
        test(x - 1)
\end{python3code}

\subsection{Summary}
\label{subsec:adv.sum}

\begin{itemize}
  \item A function may call itself. This is called recursion.
  \item Always need a base case, or the function will keep calling itself
    forever.
\end{itemize}

\pagebreak

\section{Exercises}

%\begin{warningbox}{Requirements}
%  Remember to adhere to PEP 8, PEP 257, and the boilerplate code requirement.
%\end{warningbox}


\begin{warningbox}{Install}
  Please have the \pythoninline{matplotlib} module installed on your computer
  before next lab. If you installed Anaconda Python, you already have it. Please
  come see a TA if you are having problems with the installation!
\end{warningbox}


\begin{ex}[geometry.py, triangle.py, regularpolygons.py, rectangle.py, circle.py]
  Write a Python script that lets the user do some basic
  geometry calculations. Your output should look something like the following.

  The majority of the output must be calculated by your program using a function
  for each kind of calculation. \emph{The output given below is just an example.
  Your program needs to perform the calculations for the numerical values read
  from the keyboard.}

  \begin{verbatimcode}
Welcome to Wile E. Coyote's Geometry Calculator!

Enter height of a rectangle >>> 2
Enter width of a rectangle >>> 3

The area of a rectangle with height 2 and width 3 is 6.
The perimeter of a rectangle with height 2 and width 3 is 10.
The length of the diagonal of a rectangle with height 2 and width 3 is 3.605551.

Enter the radius of a circle >>> 2

The area of a circle with radius 2.00 is 12.57.
The circumference of a circle with radius 2.00 is 12.57.

Enter the height of a right triangle >>> 1
Enter the base of a right triangle >>> 1

The area of a right triangle with height 1.00 and base 1.00 is 0.50.
The perimeter of a triangle with height 1.00 and base 1.00 is 3.41.

Enter the number of sides of a regular polygon as >>> 8
Enter the length of the side of a regular polygon >>> 5

The exterior angle of a regular polygon with 8 sides is 45.00 degrees.
The interior angles of a regular polygon with 8 sides sum to 1080.00 degrees.
Each interior angle of a regular polygon with 8 sides is 135.00 degrees.
The area of a regular polygon with 8 sides each 5.00 long is 120.71.
  \end{verbatimcode}

  \begin{center}
    \bfseries Remember that coding the input part of the program should be your
    \emph{last} step. It is a beginner's mistake to start with that -- you want
    the logic of your functions to be correct before doing any of the input.
    Test your functions with hard-coded values.
  \end{center}

  Please split your program into multiple files. Please create a module each for
  the rectangle functions, the circle functions, the triangle functions, and the
  polygon functions. For each of the modules, use boilerplate code to test the
  functions. Here's an example:

  \begin{python3code}
# Module square (square.py)

def area(side_length):
    """Calculates the area of a square given the length of a side."""
    return side_length ** 2

def perimeter(side_length):
    """Calculates the perimeter of a square given the length of a side."""
    return 4 * side_length

def main():
    side_len = 2
    print('A square with side length {:.2f} has area {:.2f}.'.format(
        side_len, area(side_len)))
    print('A square with side length {:.2f} has perimeter {:.2f}.'.format(
        side_len, perimeter(side_len)))

if __name__ == '__main__':
    main()
  \end{python3code}

  Then, write a script \texttt{geometry.py} that gives the user interface as
  seen in the example above.

  Your program must define the following functions. It is your task to figure
  out the appropriate arguments and return values as well as the code of these
  functions. Notice that the triangle is a \emph{right triangle} where height
  and base are the two legs joined by the 90 degree angle.
  \begin{multicols}{2}
    \begin{itemize}
      \item Module \texttt{triangle}:
        \begin{itemize}
          \item \pythoninline{area}
          \item \pythoninline{hypotenuse}
          \item \pythoninline{perimeter}
        \end{itemize}
      \item Module \texttt{regularpolygons}:
        \begin{itemize}
          \item \pythoninline{exterior_angle}
          \item \pythoninline{interior_angle}
          \item \pythoninline{area}
        \end{itemize}
      \item Module \texttt{rectangle}:
        \begin{itemize}
          \item\pythoninline{area}
          \item \pythoninline{perimeter}
          \item \pythoninline{diagonal}
        \end{itemize}
      \item Module \texttt{circle}:
        \begin{itemize}
          \item \pythoninline{area}
          \item \pythoninline{circumference}
        \end{itemize}
    \end{itemize}
  \end{multicols}

  \emph{Regular Polygons}

  The exterior angle of a regular $n$-polygon is $\frac{360}{n}$, where $n$ is
  the number of sides. The angle is in degrees.

  The sum of the interior angles of a regular $n$-polygon is $180(n-2)$, where the
  angle is in degrees.

  To get one particular interior angle of a regular $n$-polygon, divide by the
  number of angles $n$ to get $\frac{180(n-2)}{n}$.

  Finally, to find the area of a regular $n$-polygon, use the following formula:
  \[ \frac{s^2 \times n}{4 \times \tan\left( \frac{\pi}{n} \right)}, \]
  where $s$ is the length of a side and $n$ is the number of sides.
\end{ex}

\begin{ex}[sum.py] Write a recursive function that computes the sum of all the
  numbers from 1 to $n$, where $n$ is the given parameter.
\end{ex}

\begin{ex}[peasants.py] \hfill
  \begin{enumerate}[(a)]
    \item Russian peasant multiplication is a method of multiplying integers in
      which you keep halving one factor while
      doubling the other factor until one of the factors is 1. For example:
      \begin{IEEEeqnarray*}{RCRCL}
        &\quad& 8 &\quad\times\quad& 38 \\
        = && 4 &\quad\times\quad& 76 \\
        = && 2 &\quad\times\quad& 152 \\
        = && 1 &\quad\times\quad& 304
      \end{IEEEeqnarray*}
      Hence, $8 \times 38 = 304$.

      This, of course, becomes a bit more involved if the integer that you keep
      halving is not a power of two. For example, imagine $38$ is the number
      that keeps getting halved:
      \begin{IEEEeqnarray*}{RCRCLCLCL}
        && 8 &\times& 38 \\
        = &\quad& 16 &\times& 19 \\
        = && 32 &\times& 9 &+& 16      &\qquad& \text{$\nicefrac{19}{2} = 9$ with remainder $1$} \\
        = && 64 &\times& 4 &+& 16 + 32 &\qquad& \text{$\nicefrac{9}{2} = 4$ with remainder $1$} \\
        = && 128 &\times& 2 &+& 16 + 32 \\
        = && 256 &\times& 1 &+& 16 + 32 \\
        = && 256 && &+& 16 + 32 \\
        = && 304
      \end{IEEEeqnarray*}

      This is all based on the following recursive definition of multiplication:

      \[ x \times y = \begin{cases}
          \frac{x}{2} \times (2 \times y) & \text{if $x$ is even} \\
          \frac{x-1}{2} \times (2 \times y) + y & \text{if $x$ is odd}
      \end{cases} \]

      Write a \emph{non-recursive} function that implements Russian
      peasant multiplication using a \pythoninline{while} loop. Call this
      function \pythoninline{multiply()}.

    \item The Russian peasant method can also be applied to exponentiation (also
      only for integers):

      \[ x^y = \begin{cases}
          \left( x^2 \right)^{\nicefrac{y}{2}} & \text{if $y$ is even} \\
          x \left( x^2 \right)^{\nicefrac{(y-1)}{2}} & \text{if $y$ is odd} \\
      \end{cases} \]

      This method is also called exponentiation by squaring or the
      square-and-multiply method.

      Write a \emph{recursive} function called \pythoninline{expo()} that
      implements this.

  \end{enumerate}

\end{ex}

\begin{ex}[pascal.py] \hfill

  In Pascal's triangle, each number is the sum of the two numbers directly above
  it. The left and right ends of the triangle always consists of $1$s.

  This recursive rule produces the following triangle:

  \begin{center}
\def\N{9}
\tikz[x=0.75cm,y=0.5cm, 
  pascal node/.style={font=\small}, 
  row node/.style={font=\footnotesize, anchor=west, shift=(180:1)}]
  \path  
    \foreach \n in {0,...,\N} { 
      (-\N/2-1, -\n) node  [row node/.try]{Row \n:}
        \foreach \k in {0,...,\n}{
          (-\n/2+\k,-\n) node [pascal node/.try] {%
            \pgfkeys{/pgf/fpu}%
            \pgfmathparse{round(\n!/(\k!*(\n-\k)!))}%
            \pgfmathfloattoint{\pgfmathresult}%
            \pgfmathresult%
          }}};
  \end{center}

  Write a recursive function \pythoninline{pascal(n, k)} that finds the $k$th
  value of the $n$th row by using the sum of the numbers directly above it.
  We start counting at $0$ for both $n$ and $k$.

  Since we know that the left and right end of the triangle are all $1$s, we
  know that for every row $n$:
  \begin{IEEEeqnarray*}{RCLCL}
    \pas(n, 0) &=& 1 &\qquad& \text{left end} \\
    \pas(n, n) &=& 1 &\qquad& \text{right end}
  \end{IEEEeqnarray*}

  If for example I want to know the $2$nd entry of the $4$th row:

  \begin{IEEEeqnarray*}{RCLCL}
    \pas(4, 2) &=& \pas(3, 1) + \pas(3, 2)  \\
    &=& \pas(2, 0) + \pas(2, 1) + \pas(2, 1) + \pas(2, 2) \\
    &=& 1 + \pas(1, 0) + \pas(1, 1) + \pas(1, 0) + \pas(1, 1) + 1 \\
    &=& 1 + 1 + 1 + 1 + 1 + 1 \\
    &=& 6
  \end{IEEEeqnarray*}
\end{ex}

\section{Submitting}

You should submit your code as a tarball. It should contain all files
used in the exercises for this lab. The submitted file should be named
\begin{center}
  \texttt{cse107\_firstname\_lastname\_lab3.tar.gz}
\end{center}

\begin{center}
  \textbf{Upload your tarball to Canvas before the deadline.}
\end{center}

\listoftheorems

\end{document}
