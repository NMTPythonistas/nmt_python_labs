% LAB 10: Car Project
%
% CSE/IT 107: Introduction to Programming
% New Mexico Tech
%
% Prepared by Russell White and Christopher Koch
% Spring 2015
\documentclass[11pt]{cselabheader}
\usepackage{caption}

%%%%%%%%%%%%%%%%%% SET TITLES %%%%%%%%%%%%%%%%%%%%%%%%%
\fancyhead[R]{Lab 10: Car Project}
\title{Lab 10: Car Project}

\begin{document}

\maketitle

\hrule

\begin{quotation}
  ``When I see a bird that walks like a duck and swims like a duck and quacks like
  a duck, I call that bird a duck.''
\end{quotation}
\begin{flushright}
  --- James Whitcomb Riley, Duck Enthusiast
\end{flushright}



\begin{quotation}
``Imagination is more important than knowledge.''
\end{quotation}
\begin{flushright}
  --- Albert Einstein
\end{flushright}

\begin{quotation}
``The limits of my language mean the limits of my world.''
\end{quotation}
\begin{flushright}
  --- Ludwig Wittgenstein
\end{flushright}

\hrule


\section{Car Project}

\begin{warningbox}{Cars Are Hard}
  \emph{Warning}: The following exercise was designed with little regard for
  how cars and physics actually work in the real world. Sorry.
\end{warningbox}

\begin{ex}[car.py, tire.py, engine.py, race.py]
  For this project, you will be submitting four files:

  \pythoninline{car.py} will contain the definition of \pythoninline{class Car}. A
  Car will have an \pythoninline{Engine} and a \pythoninline{Tire}  (representing all 4)
  as attributes. Additionally, each car should have an attribute representing its name.
  Additionally, a Car should have a method \pythoninline{travel_time(self, time)} that returns
  the distance that the Car travels in the time given as well as a method
  \pythoninline{travel_dist(self, dist)} that returns the time it takes for the Car to travel
  the given distance. Both methods should assume that
  the car starts at a standstill and accelerates throughout the travel time. The results
  should take into account the attributes of the Car's Tire and Engine.
  
  \pythoninline{tire.py} will contain the definition of \pythoninline{class Tire}. Each Tire
  will have two attributes: traction and radius. The traction of the wheel indicates the
  maximum ideal acceleration the wheel can provide. If the car's ideal acceleration
  would be above this number, then the actual acceleration
  is instead the average of this number and the
  ideal acceleration. The radius of the tire determines how far it causes the
  car to travel per rotation. This is important because Engine strength is
  measured in rotations per second while traction is measured in meters
  per second squared and radius is measured in meters.
  
  \pythoninline{engine.py} will contain the definition of
  \pythoninline{class Engine}. Each Engine will have two attributes:
  max speed and acceleration. Max speed is in terms of tire rotations per
  second. Acceleration is a value between $0$ and $1$ that indicates how much
  closer the current speed gets to the max speed each second. Each second, the
  new engine speed is determined by the following formula:
  $$newSpeed = oldSpeed + acceleration \cdot \left(maxSpeed - oldSpeed\right)$$
  Since engine speed is measured in rotations per second, the actual
  acceleration of the car must take into account the engine's speed, the tire's
  traction, and the tire's radius to determine how far it travels each second.
  Because of the tire's traction, it is possible that the car is not
  experiencing the ideal speed that might be indicated by
  $engineSpeed \cdot tireCircumference$. Thus, the car's speed cannot be purely
  calculated from the engine's speed each second.

  \pythoninline{race.py} will contain your \pythoninline{main()} function and will
  manipulate the classes from the other files. The program should call the
  constructors for each of the classes described above in order to create
  several complete cars, then print out data comparing the performance of the
  cars. The data displayed should include every attribute about every car,
  followed by a table comparing the performance of each car in a trial
  comparing either how far each car travels in a given time or how long each
  car takes to travel a given distance. These should use the relevant methods
  described above. You should ask the user whether they want a table of distances
  gone in fixed times or times taken to go fixed distances. You should be able
  to compare 1, 2, 3, or 4 cars in one table. Table~\ref{gofast} shows the
  data you need to show when comparing distances given a fixed time. You should be
  able to make a similar table where the first column is distances (likely in
  intervals of 100) and the others are how long the cars took to reach those
  distances.
  
  \begin{center}
    \begin{tabular}{crrrr}
    \toprule
    Car Name & Traction & Tire Radius & Max Speed & Acceleration\\
    \midrule
    Rabbit & 0 & 0.5 & 1000 & 0.1\\
    Hare & 100 & 0.1 & 500 & 1.0\\
    \bottomrule
    \end{tabular}
    \captionof{table}{Data to print out about cars.}
  \end{center}
  
  \begin{center}
    \begin{tabular}{crr}
    \toprule
    Time & Rabbit & Hare\\
    \midrule
    0 & 0.00 & 0.00\\
    1 & 0.00 & 0.00\\
    2 & 157.08 & 207.08\\
    3 & 534.07 & 517.70\\
    4 & 1148.25 & 831.86\\
    5 & 1995.54 & 1146.02\\
    6 & 3062.44 & 1460.18\\
    7 & 4331.90 & 1774.34\\
    8 & 5786.12 & 2088.50\\
    9 & 7407.85 & 2402.65\\
    10 & 9180.95 & 2716.81\\
    \bottomrule
    \end{tabular}
    \captionof{table}{Data to display about how far cars go in given times.}
    \label{gofast}
  \end{center}
  
  When simulating the cars, you should use a time delta of 1 second. This means
  that you simulate time in chunks of one second at a time. Thus, a car should
  be considered to have the same speed throughout the entirety of each second.
  This can result in some inaccurate results (such as each car always
  traveling a distance of 0 for the first second), but is fine for our
  purposes. As an example of how a car's travel might work, see
  Table~\ref{carexample}, which is a breakdown of the stats of a car at each
  second of a 10 second trip. The car's tires have a radius of $\frac{1}{\pi}$
  (giving a circumference of 2) and a traction of 10. The car's engine has a
  max speed of 100 and an acceleration of 0.5.
  
  \begin{center}
    \begin{tabular}{crrrr}
    \toprule
    Time $(s)$ & Distance $(m)$ & Car Speed $(\frac{m}{s})$ &
    Ideal Car Speed $(\frac{m}{s})$ & Engine Speed $(\frac{rot}{s})$\\
    \midrule
    0  &    0.00 &   0.00 &   0.00 &  0.00\\
    1  &    0.00 &  55.00 & 100.00 & 50.00\\
    2  &   55.00 & 107.50 & 150.00 & 75.00\\
    3  &  162.50 & 146.25 & 175.00 & 87.50\\
    4  &  308.75 & 171.88 & 187.50 & 93.75\\
    5  &  480.63 & 187.81 & 193.75 & 96.88\\
    6  &  668.44 & 196.88 & 196.88 & 98.44\\
    7  &  865.31 & 198.44 & 198.44 & 99.22\\
    8  & 1063.75 & 199.22 & 199.22 & 99.61\\
    9  & 1262.97 & 199.61 & 199.61 & 99.80\\
    10 & 1462.58 & 199.80 & 199.80 & 99.90\\
    \bottomrule
    \end{tabular}
    \captionof{table}{Example of a Car's Performance}
    \label{carexample}
  \end{center}
  
  It should be noted that the Distance column of Table~\ref{carexample}
  represents the return value of Car's
  \pythoninline{distance_time} method when given the
  corresponding Time column as an argument.
\end{ex}

\pagebreak
\section{Submitting}
You should submit your code as a tarball. It should contain all files
used in the exercises for this lab. The submitted file should be named
\begin{center}
  \texttt{cse107\_firstname\_lastname\_lab10.tar.gz}
\end{center}

\begin{center}
  \textbf{Upload your tarball to Canvas.}
\end{center}

\listoftheorems

\end{document}
