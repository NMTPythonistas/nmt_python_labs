% LAB 10: Classes
% 
% CSE/IT 107: Introduction to Programming
% New Mexico Tech
% 
% Prepared by Russell White and Christopher Koch
% Spring 2015
\documentclass[11pt]{cselabheader}
\usepackage{caption}

%%%%%%%%%%%%%%%%%% SET TITLES %%%%%%%%%%%%%%%%%%%%%%%%%
\fancyhead[R]{Lab 10: Classes}
\title{Lab 10: Classes}

\begin{document}

\maketitle

\hrule
\begin{quotation}
``Holy shit, you geeks are badass.''
\end{quotation}
\begin{flushright}
  --- Pam (\emph{Archer})
\end{flushright}

\hrule

\pagebreak
\section{Introduction}


\pagebreak

\section{Classes and Objects}
\label{sec:classes}
In Python, classes are a way to group together functions and attributes
that are closely related to one another. A \lstinline{class} is defined
similarly to a function, though it uses the \lstinline{class} keyword rather
than \lstinline{def}.

\begin{pyconcode}
>>> class Test:
...     i = 5
...     def hi():
...         print('Hello world.')
... 
>>> Test.i
5
>>> Test.hi()
Hello world.
\end{pyconcode}

Once a class has been declared, an \emph{Object} or
\emph{Instance} of that class can be created. An object is an independent
copy of the class, with all of the same attributes and functions that were
declared with the class that the object was created from. This is the
primary use for classes, as it allows for you to easily create a group of
objects that store the same kind of data.

When doing this, it is important to declare a special function called
\lstinline{__init__} inside the class. This function is called when
creating the new class and can be used to initialize values for the
new object:

\begin{python3code}
class Test:
    def __init__(self):
        self.x = 10

    def set(self, x):
        self.x = x

a = Test()
b = Test()
print(a.x, b.x)
a.set(20)
print(a.x, b.x)
\end{python3code}

\begin{pyconcode}
10 10
20 10
\end{pyconcode}

Note the use of the variable \lstinline{self} in the function declarations.
This variable should be first in any function declared that needs to be used on
a function level. This variable refers to the object that the function is
being called on. For example, in line 11, \lstinline{self} would refer to
\lstinline{a}. Note that \lstinline{a.set(20)} is equivalent to
\lstinline{Test.set(a, 20)}. The former is preferred because it is shorter and
more readable. Any variable referred to with \lstinline{self.variablename} will
stay with the object, while other variables will stay local to the function.

\lstinline{__init__} can be passed additional arguments in addition to
\lstinline{self} so that the new objects can be initialized differently.

The special function \lstinline{__str__(self)} can be defined to override
how the object is displayed when printed. By default, printing an object will
display something like this:

\begin{pyconcode}
>>> class Test:
...     pass
... 
>>> a = Test()
>>> print(a)
<__main__.Test object at 0x7f2b806cc9e8>
\end{pyconcode}

However, if we define the function, it will instead print whatever we want it
to. \lstinline{__str__} should always return a string.

\begin{python3code}
class Test:
    def __init__(self):
        self.x = 5

    def __str__(self):
        return 'x = {}'.format(self.x)

a = Test()
print(a)
\end{python3code}

Try running the above code. It should print out \lstinline{x = 5}. Overriding
\lstinline{__str__} makes our output code far simpler when printing objects.

\section{Inheritance}
\label{sec:inheritance}


\clearpage
\section{Exercises}
\label{sec:ex}



\pagebreak
\section{Submitting}
You should submit your code as a tarball. It should contain all files
used in the exercises for this lab. The submitted file should be named
\begin{center}
  \texttt{cse107\_firstname\_lastname\_lab10.tar.gz}
\end{center}

\begin{center}
  \textbf{Upload your tarball to Canvas.}
\end{center}

\listoftheorems


\end{document}
