% LAB 9: Classes
%
% CSE/IT 107: Introduction to Programming
% New Mexico Tech
%
% Prepared by Russell White and Christopher Koch
% Spring 2015
\documentclass[11pt]{cselabheader}
\usepackage{caption}

%%%%%%%%%%%%%%%%%% SET TITLES %%%%%%%%%%%%%%%%%%%%%%%%%
\fancyhead[R]{Lab 9: Classes}
\title{Lab 9: Classes}

\begin{document}

\maketitle
\pagenumbering{roman}
\hrule

\begin{quotation}
``Who are you? How did you get in my house?''
\end{quotation}
\begin{flushright}
  --- \href{https://xkcd.com/163/}{Donald Knuth}
\end{flushright}

\hrule

\section{Introduction}
This week, we will have a brief introduction to Object Oriented Programming by
way of Python's classes.

\pagebreak
\tableofcontents
\pagebreak
\pagenumbering{arabic}

\section{Modules}
\label{sec:modules}

A program that is saved in a text file and then run with Python is called a
\emph{script}. As your programs get longer, you may want to split them into
multiple files for easier legibility, maintenance, or abstraction. To do this,
Python supports a way to put definitions (of functions) in a file and use them
in another script or in the interpreter. A file like this is called a
\emph{module}. An example for such a module is the math module we have used
previously. The definitions from a module can be \emph{imported} into other
scripts.

A module is just a file with Python statements in it. A module takes the name of
the file that it is in. For example, imagine we have a file
\texttt{circlemath.py} in our current working directory:

\begin{listing}[H]
\vspace{-0.5em}
\begin{python3code}
import math # importing the math module

def area(radius):
    """Find and return the area of a circle (float) given the radius"""
    return math.pi * radius**2

def circumference(radius):
    """Find and return the circumference of a circle (float) given the radius"""
    return 2 * math.pi * radius
\end{python3code}
\vspace{-1em}
\caption{\texttt{circlemath.py}}
\vspace{-0.5em}
\end{listing}

Then, open the interpreter and import the module:

\begin{pyconcode}
>>> import circlemath
>>> circlemath.circumference(5) # have to use modulename.attributename
31.41592653589793
>>> circlemath.area(5)
78.53981633974483
>>> circlemath.__name__
'circlemath'
\end{pyconcode}

Every module has a \pythoninline{__name__} attribute that contains the name of
the module, except in one special case that we will see soon.

There are other kinds of import statements with different effects:

\begin{pyconcode}
>>> from circlemath import area
>>> area(5) # can just use the attribute imported without modulename.
78.53981633974483
>>> circumference(5) # not defined, because not imported
Traceback (most recent call last):
  File "<input>", line 1, in <module>
NameError: name 'circumference' is not defined
\end{pyconcode}

\begin{pyconcode}
>>> from circlemath import area, circumference # use commas to separate multiple
>>> area(5)
78.53981633974483
>>> circumference(5)
31.41592653589793
\end{pyconcode}

\begin{pyconcode}
>>> from circlemath import * # import all definitions made
>>> area(5)
78.53981633974483
>>> circumference(5)
31.41592653589793
\end{pyconcode}

\begin{pyconcode}
>>> from circlemath import area as carea
>>> carea(5)
78.53981633974483
\end{pyconcode}

The last option is rather frowned upon; we would prefer you to use
\pythoninline{circlemath.area} instead as it is more descriptive.

You may also rename a module upon importing it:

\begin{pyconcode}
>>> import circlemath as cmath
>>> cmath.area(5)
78.53981633974483
\end{pyconcode}


\begin{infobox}{Convention (PEP 8)}
  Always put \pythoninline{import} statements at the beginning of a file in the
  following order:
  \begin{enumerate}
    \item Built-in standard library modules
    \item Third-party modules (e.g. matplotlib)
    \item Self-written modules
  \end{enumerate}
  Put a blank line between each group of imports.
\end{infobox}

\subsection{Using Modules as Scripts and Boilerplate}

If there is code in your module that is not a function definition, Python will
run it just once when the module is imported. For example, take the file
\texttt{mid.py}:

\begin{listing}[H]
  \vspace{-0.5em}
\begin{python3code}
def midpoint(a, b):
    """Find and return the midpoint of two numbers."""
    return (a+b)/2

print('Midpoint of {} and {} is {:.2f}.'.format(5, 10, midpoint(5, 10)))
\end{python3code}
  \vspace{-1em}
  \caption{\texttt{mid.py}}
  \vspace{-0.5em}
\end{listing}

\begin{pyconcode}
>>> import mid
Midpoint of 5 and 10 is 7.50.
\end{pyconcode}

The same will happen if you run the module like a script:

\begin{bashcode}
$ python3 mid.py
Midpoint of 5 and 10 is 7.50.
\end{bashcode}

However, this can be rather annoying to deal with, for example if you wrote a
script with a bunch of functions a while ago and you just want to use the
functions you wrote, but do not want the other code to run when you're using
them -- you just want the functions. You can achieve this!

When you run a module as a script, the modules \pythoninline{__name__} attribute
is not set to the module's name, but to \pythoninline{"__main__"}. Hence, the
following code in \texttt{mid.py} will do the following:

\begin{listing}[H]
  \vspace{-0.5em}
\begin{python3code}
def midpoint(a, b):
    """Find and return the midpoint of two numbers."""
    return (a+b)/2

print(__name__) # just for demonstration, do not put this in real programs

if __name__ == "__main__":
    print('Midpoint of {} and {} is {:.2f}.'.format(5, 10, midpoint(5, 10)))
\end{python3code}
  \vspace{-1em}
  \caption{\texttt{mid.py}}
  \vspace{-0.5em}
\end{listing}

\begin{pyconcode}
>>> import mid
mid
\end{pyconcode}

\begin{bashcode}
$ python3 mid.py
__main__
Midpoint of 5 and 10 is 7.50.
\end{bashcode}

However, sometimes you may want to import the module \emph{and} run it as if it
were a script. To enable us to do so, we usually put the script code in a
function called \pythoninline{main()}:

\begin{listing}[H]
  \vspace{-0.5em}
\begin{python3code}
def midpoint(a, b):
    """Find and return the midpoint of two numbers."""
    return (a+b)/2

def main():
    print('Midpoint of {} and {} is {:.2f}.'.format(5, 10, midpoint(5, 10)))

if __name__ == "__main__":
    main()
\end{python3code}
  \vspace{-1em}
  \caption{\texttt{mid.py} -- perfect example}
  \vspace{-0.5em}
\end{listing}

\begin{pyconcode}
>>> import mid
>>> mid.main() # if we want to
Midpoint of 5 and 10 is 7.50.
\end{pyconcode}

\begin{bashcode}
$ python3 mid.py
Midpoint of 5 and 10 is 7.50.
\end{bashcode}

We call this combination of the if-statement and the
\pythoninline{main()} function \emph{boilerplate code}.

\subsection{The \protect\pythoninline{dir()} Function}

Use the \pythoninline{dir()} function to get a list of every attribute --
variables, functions, and other things you do not know about yet -- that is part
of a module. For example for the circlemath module we wrote earlier:

\begin{pyconcode}
>>> import circlemath
>>> dir(circlemath)
['__name__', 'area', 'circumference']
>>> import mid
>>> dir(mid)
['__name__', 'midpoint']
>>> import math
>>> dir(math)
['__doc__', '__file__', '__loader__', '__name__', '__package__', '__spec__', 'acos',
'acosh', 'asin', 'asinh', 'atan', 'atan2', 'atanh', 'ceil', 'copysign', 'cos', 'cosh',
'degrees', 'e', 'erf', 'erfc', 'exp', 'expm1', 'fabs', 'factorial', 'floor', 'fmod',
'frexp', 'fsum', 'gamma', 'hypot', 'isfinite', 'isinf', 'isnan', 'ldexp', 'lgamma',
'log', 'log10', 'log1p', 'log2', 'modf', 'pi', 'pow', 'radians', 'sin', 'sinh', 'sqrt',
'tan', 'tanh', 'trunc']
\end{pyconcode}

\subsection{Summary}
\label{subsec:modules.sum}

\begin{itemize}
  \item Syntax:

    \begin{python3code}
import modname1, modname2 # use: modname1.attributename
import modname3 as othermodname # use: othermodname.attributename
from modname import attribute1, attribute2 # use: attribute1, attribute2
from modname import * # use: attributename
from modname import attribute as otherattribute # use: otherattribute
    \end{python3code}

    Attributes can be functions defined in the module or variables defined in
    the module. A special attribute is \pythoninline{__name__}, which contains
    the name of the module unless the module is executed as a script.

  \item Every script is a module whose name is the filename of the script.

  \item If a module is imported, its \pythoninline{__name__} attribute is a
    variable that contains the module name. If a module is run as a script, its
    \pythoninline{__name__} attribute contains \pythoninline{"__main__"}. With
    this, you can have code that only runs if a module is run like a script.

    We call that boilerplate code:

    \begin{python3code}
def main():
    # stuff that only runs if module is run as a script

if __name__ == "__main__":
    main()
    \end{python3code}

  \item Use \pythoninline{dir()} to get a list of every attribute -- every
    variable and function (and other things that you do not know about yet) --
    that is part of a particular module.

  \item A good resource:

    \begin{center}
      \url{https://docs.python.org/3.4/tutorial/modules.html}
    \end{center}

\end{itemize}


\section{Classes and Objects}
\label{sec:classes}
In Python, classes are a way to group together functions and attributes
that are closely related to one another. A \pythoninline{class} is defined
similarly to a function, though it uses the \pythoninline{class} keyword rather
than \pythoninline{def}.

\begin{pyconcode}
>>> class Test:
...     i = 5
...     def hi():
...         print('Hello world.')
...
>>> Test.i
5
>>> Test.hi()
Hello world.
\end{pyconcode}

Once a class has been declared, an \emph{Object} or
\emph{Instance} of that class can be created. An object is an independent
copy of the class, with all of the same attributes and functions that were
declared with the class that the object was created from. This is the
primary use for classes, as it allows for you to easily create a group of
objects that store the same kind of data.

When doing this, it is important to declare a special function called
\pythoninline{__init__} inside the class. This function is called when
creating the new class and can be used to initialize values for the
new object:

\begin{python3code}
class Test:
    def __init__(self):
        self.x = 10

    def set(self, x):
        self.x = x

a = Test()
b = Test()
print(a.x, b.x) #10 10
a.set(20)
print(a.x, b.x) #20 10
\end{python3code}

Note the use of the variable \pythoninline{self} in the function declarations.
This variable should be first in any function declared that needs to be used on
a function level. This variable refers to the object that the function is
being called on. For example, in line 11, \pythoninline{self} would refer to
\pythoninline{a}. Note that \pythoninline{a.set(20)} is equivalent to
\pythoninline{Test.set(a, 20)}. The former is preferred because it is shorter
and more readable. Additionally, a function that refers operates on an instance
of the class (that is, has \pythoninline{self} as the first argument) are called
methods rather than functions.A ny variable referred to with
\pythoninline{self.variablename} will stay with the object, while other
variables will stay local to the function.

\pythoninline{__init__} can be passed additional arguments in addition to
\pythoninline{self} so that the new objects can be initialized differently.

The special method \pythoninline{__str__(self)} can be defined to override
how the object is displayed when printed. By default, printing an object will
display something like this:

\begin{pyconcode}
>>> class Test:
...     pass
...
>>> a = Test()
>>> print(a)
<__main__.Test object at 0x7f2b806cc9e8>
\end{pyconcode}

However, if we define the method, it will instead print whatever we want it
to. \pythoninline{__str__} should always return a string.

\begin{python3code}
class Test:
    def __init__(self):
        self.x = 5

    def __str__(self):
        return 'x = {}'.format(self.x)

a = Test()
print(a)
\end{python3code}

Try running the above code. It should print out \pythoninline{x = 5}. Overriding
\pythoninline{__str__} makes our output code far simpler when printing objects.

\section{Inheritance}
\label{sec:inheritance}
A class can be built from another class. This is commonly done when multiple
classes should logically be different but share common functionality. A class
that inherits from another class is considered to be a subclass or child of
its parent class. The subclass gains all the functions and attributes of
its parent class.

An example
might be that both Books and Articles could be considered Publications. Both
would have a title and an author, but only the book would have a chapter count.

\begin{python3code}
class Publication:
    def __init__(self, title, author):
        self.title = title
        self.author = author

    def __str__(self):
        return '{} by {}'.format(self.title, self.author)

class Book(Publication):
    def __init__(self, title, author, chapters):
        Publication.__init__(self, title, author)
        self.chapters = chapters

class Article(Publication):
    def __init__(self, title, author, magazine):
        Publication.__init__(self, title, author)
        self.magazine = magazine

    def __str__(self):
        return '{} by {}, published in {}'.format(self.title, self.author,
            self.magazine)

b = Book('Title', 'Author', 100)
a = Article('Other Title', 'Other Author', 'Some Magazine')

print(b) #Title by Author
print(a) #Other Title by Other Author, published in Some Magazine
\end{python3code}

Note that \pythoninline{Article} creates its own version of the
\pythoninline{__str__} method, while \pythoninline{Book} does not. This means
that \pythoninline{Book} simply uses the method that it inherits from
\pythoninline{Publication}.

Also note that each of the subclasses call the constructors of the parent class.
This is required to properly initialize the attributes of the object. Though it
is not strictly required, doing otherwise would require duplicating the
contents of the parent class's constructor in the subclass. A class can inherit
from multiple other classes. In that case, it would be expected to call the
constructors for both of its parent classes.

Sometimes you will want to check if an object belongs to a specific class. This
can be done with the \pythoninline{isinstance} function. This function will
return \pythoninline{True} if the given object is an instance of the given class
or of a subclass of that class.

\begin{python3code}
class A:
    pass
class B:
    pass
class C(A):
    pass
class D(C, B):
    pass

print(isinstance(A(), A)) #True
print(isinstance(B(), A)) #False
print(isinstance(C(), A)) #True
print(isinstance(D(), B)) #True
\end{python3code}

\section{Iteration}
\label{sec:iter}
By defining a special method, we can make an object of any class that
we create iterable. This function is \pythoninline{__iter__(self)}.
This function is used by anything that uses
iterables (such as \pythoninline{for} loops). \pythoninline{__iter__} is called
when the loop starts and then uses the \pythoninline{yield} keyword to return
each successive element for the loop. \pythoninline{yield} works almost exactly
like \pythoninline{return} except it does not end the function. Instead the function
is ``paused''
until the next value is requested, at which point execution continues until the
next \pythoninline{yield} is encountered. The loop ends when the function raises
and exception called \pythoninline{StopIteration}.
Using this knowledge we can write our
own version of \pythoninline{range()}:

\begin{python3code}
class MyRange:
    def __init__(self, start, end):
        self.start = start
        self.end = end

    def __iter__(self):
        current = self.start
        while current < self.end:
            yield current
            current += 1
        raise StopIteration

r = MyRange(1, 10)
for i in r:
    print(i) #prints 1, 2, 3, ..., 9
\end{python3code}

With a bit more complex logic in our \pythoninline{__iter__} we can do some
pretty complicated things with \pythoninline{for} loops!

\clearpage
\section{Exercises}
\label{sec:ex}

\begin{ex}[library.py]
    Create a module called \pythoninline{library}. This module should include
    two classes as well as methods for each. The provided file
    \pythoninline{library_test.py} will use the contents of
    \pythoninline{library.py}. Do not edit \pythoninline{library_test.py} for
    your submission, though you are free to comment out lines using parts of
    \pythoninline{library.py} that you have not implemented yet. The contents of
    \pythoninline{library.py} should be as follows:

    \begin{description}
    \item[class Library] A \pythoninline{Library} object stores multiple
        \pythoninline{Book}
        objects and has methods to return statistics about those books. When
        printed, the object should list all of the books it contains in
        alphabetical order by author (as listed. You do not need to split into
        first and last names).
        \begin{description}
        \item[def add\_book(self, book)] This method takes in either a line of
            text or a book object. If the input is a line of text, then it
            creates a new \pythoninline{Book} object out of it. The line of
            text will be formatted as are the lines in \pythoninline{library.txt}
            and should be parsed as needed to form a \pythoninline{Book} object.
            The object (whether given or created) is then stored inside the
            library.
        \item[def get\_authors(self)] This method returns a list of all the
            authors who have works contained in the library. No author should
            appear more than once.
        \item[def get\_books\_per\_author(self)] This method returns a
            dictionary.
            Each author should be a key in the dictionary. The value for each
            author should be the number of books by that author in the library.
        \end{description}
    \item[class Book] A \pythoninline{Book} object stores the author and title
        of a book. When printed, the object should display both the title and
        author of the book. The constructor should take in two arguments:
        the title of the book and the author of the book.
    \end{description}

    You are free (and encouraged) to include any additional methods that you may
    need to accomplish the tasks given.
\end{ex}

\begin{ex}[myturtle.py]
    Write a class that inherits from \pythoninline{turtle.Turtle}. Overwrite
    the \pythoninline{forward}, \pythoninline{backward}, \pythoninline{left},
    and \pythoninline{right} methods so that, if given a negative value, they
    cause no movement to occur. Add a new method \pythoninline{regular_polygon}
    that takes as arguments a number of sides and a length for the sides, then
    draws out the regular polygon fulfilling those requirements.

    Write a program using your replacement \pythoninline{Turtle} class that
    creates multiple objects of that class and tests out each of the methods
    you have written.
\end{ex}

\begin{ex}[tree.py]
    Write a class that implements a simple binary tree data structure. That is a
    tree that consists of a group of nodes, each of which has an optional
    ``left'' and ``right'' child. The class needs to include the following:
    \begin{description}
    \item[class Tree] The class used to create objects representing trees. Each
        object will also represent one of the nodes of the tree. The constructor
        should take in one argument: the contents of that node. The contents can
        be anything and are not important for the scope of this exercise. The
        constructor should have two optional arguments: left and right trees. If
        given, these will be the left and right children of the new node. This
        allows for the joining of two trees into a single one with a new parent
        node.
        \begin{description}
        \item[self.left] The left child of the node. May be
            \pythoninline{None} or a \pythoninline{Tree} object.
        \item[self.right] The right child of the node. May be
            \pythoninline{None} or a \pythoninline{Tree} object.
        \item[self.datum] The contents of the node.
        \item[def height(self)] Returns the height of the tree that this node
            is the root of. This is equal to one more than the largest height
            of the children trees. If a child tree is \pythoninline{None} then
            its height is 0.
        \item[def add\_item(self, item)] Create a new \pythoninline{Tree} object
            and add it to the tree. The new object should be added to the child
            tree with the smallest height (remember that a \pythoninline{None}
            tree has a height of 0!). If there is a tie, favor the left tree.
        \end{description}
    \end{description}

    Additionally, you should implement \pythoninline{__iter__(self)}
    such that the tree can be used in a for loop.
    When used in a for loop, the nodes of the tree should be accessed in-order.
    This means that they should be accessed from left to right. That is, if we
    have a tree that looks like this:

\begin{verbatim}
    1
   / \
  2   3
 / \ / \
4  5 6  7
\end{verbatim}

    Then the nodes should be accessed in the order 4, 2, 5, 1, 6, 3, 7. To
    accomplish this, you will need to keep track of where in the traversal the
    loop currently is, as well as \pythoninline{raise StopIteration} when
    necessary.

    Once you have it working, you should be able to use \pythoninline{list} to
    convert your tree into a list.
\end{ex}

\pagebreak
\section{Submitting}
You should submit your code as a tarball. It should contain all files
used in the exercises for this lab. The submitted file should be named
\begin{center}
  \texttt{cse107\_firstname\_lastname\_lab9.tar.gz}
\end{center}

\begin{center}
  \textbf{Upload your tarball to Canvas.}
\end{center}

\listoftheorems

\end{document}
