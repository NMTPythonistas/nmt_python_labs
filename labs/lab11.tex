% LAB 11: Car Project
%
% CSE/IT 107: Introduction to Programming
% New Mexico Tech
%
% Prepared by Russell White and Christopher Koch
% Spring 2015
\documentclass[11pt]{cselabheader}
\usepackage{caption}

%%%%%%%%%%%%%%%%%% SET TITLES %%%%%%%%%%%%%%%%%%%%%%%%%
\fancyhead[R]{Lab 11: Car Project}
\title{Lab 11: Car Project}

\begin{document}

\maketitle

\hrule
\begin{quotation}
  ``When I see a bird that walks like a duck and swims like a duck and quacks like
  a duck, I call that bird a duck.''
\end{quotation}
\begin{flushright}
  --- James Whitcomb Riley
\end{flushright}

\hrule

\section{Introduction}


\clearpage
\section{Exercises}
\label{sec:ex}

\begin{ex}[car.py, tire.py, engine.py, race.py]
  For this project, you will be submitting four files:
  \pythoninline{car.py} will contain the definition of \pythoninline{class Car}. A
  Car will have an \pythoninline{Engine} and a \pythoninline{Tire}  (representing all 4)
  as attributes. Additionally, each car should have an attribute representing its name.
  Additionally, a Car should have a method \pythoninline{travel_time(self, time)} that returns
  the distance that the Car travels in the time given as well as a method
  \pythoninline{travel_dist(self, dist)} that returns the time it takes for the Car to travel
  the given distance.
  Both methods should assume that
  the car starts at a standstill and accelerates throughout the travel time. This value
  should take into account the attributes of the Car's Tire and Engine.
  
  \pythoninline{tire.py} will contain the definition of \pythoninline{class Tire}.
  
  \pythoninline{engine.py} will contain the definition of \pythoninline{class Engine}.

  \pythoninline{race.py} will contain your \pythoninline{main()} function and will
  manipulate the classes from the other files.
\end{ex}

\pagebreak
\section{Submitting}
You should submit your code as a tarball. It should contain all files
used in the exercises for this lab. The submitted file should be named
\begin{center}
  \texttt{cse107\_firstname\_lastname\_lab11.tar.gz}
\end{center}

\begin{center}
  \textbf{Upload your tarball to Canvas.}
\end{center}

\listoftheorems

\end{document}
