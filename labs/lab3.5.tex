% LAB 4 Review of Previous Labs
%
% CSE/IT 107: Introduction to Programming
% New Mexico Tech
%
\documentclass[11pt]{cselabheader}

%%%%%%%%%%%%%%%%%% SET TITLES %%%%%%%%%%%%%%%%%%%%%%%%%
\fancyhead[R]{Lab 4: Sequences Review}
\title{Lab 4: Sequences Review}

\begin{document}

\maketitle
\pagenumbering{roman}
\hrule

\begin{quotation}
When we describe a language, we should pay particular attention to the means that the language provides for combining simple ideas to form more complex ideas. Every powerful language has three mechanisms for accomplishing this:
\begin{description}
\item[primitive expressions,] which represent the simplest entities the language is concerned with,
\item[means of combination,] by which compound elements are built from simpler ones, and
\item[means of abstraction,] by which compound elements can be named and manipulated as units.
\end{description}
\end{quotation}
\begin{flushright}
--- Structure and Interpretation of Computer Programs
\end{flushright}
\hrule

\tableofcontents

\pagebreak
\pagenumbering{arabic}
\section{Review}
This is a short summary of the content covered in class or in the lab PDFs.

\subsection{Style and Documentation}
Remember to adhere to PEP 8 and PEP 257. See lab 3 for more details.
Code) found at
\begin{center}
  \url{https://www.python.org/dev/peps/pep-0008/}

  \url{https://www.python.org/dev/peps/pep-0257/}
\end{center}

\subsection{Boilerplate}
\label{subsec:boiler}

As you recall, the following boilerplate code is required for every lab
exercise from now on. Read lab 3 for more details and to learn why this is
needed.

\begin{infobox}{Boilerplate Requirement}
  Wrong:

  \begin{python3code}
def greeting_for(recipient):
    """Returns a string with a greeting meant for the given recipient."""
    return "Hello {}".format(recipient)

user_name = input("What is your name? ")
print(greeting_for(user_name))
  \end{python3code}

  Right:

  \begin{python3code}
def greeting_for(recipient):
    """Returns a string with a greeting meant for the given recipient."""
    return "Hello {}".format(recipient)

def main():
    user_name = input("What is your name? ")
    print(greeting_for(user_name))

if __name__ == "__main__":
    main()
  \end{python3code}
\end{infobox}


\pagebreak
\subsection{Lists}
Here are some of the methods you can use to manipulate lists.
Again, read lab 3 if any of them are unfamiliar.

\begin{table}[!ht]
  \centering
  \begin{tabular}{ll}
    \toprule
    Function/method & What it does \\
    \midrule
    \pythoninline!lst.append(x)! & appends \pythoninline!x! to \pythoninline!lst! \\
    \pythoninline!lst.insert(i, x)! & inserts \pythoninline!x! at index \pythoninline!i!
    in \pythoninline!lst! (moves other elements) \\
    \pythoninline!x = lst.pop()! & removes last element of \pythoninline!lst! and
    places it in \pythoninline!x! \\
    \pythoninline!lst.reverse()! & reverses the order of elements in list
    \pythoninline!lst! in place\\
    \pythoninline!lst.sort()! & sort \pythoninline!lst! in place \\
    \pythoninline!len(lst)! & number of elements in \pythoninline!lst! \\
    \pythoninline!sum(lst)! & sum the elements of \pythoninline!lst! (only works when
    \pythoninline!+! works between the elements)\\
    \bottomrule
  \end{tabular}
  \caption{List methods and functions, where \texttt{lst} is a variable that
  holds a list. You may also type \pythoninline{help(list)} in the console to
  get more information.}
  \label{tab:lists}
\end{table}

\subsection{Strings}
Here's a summary of the string methods and what they do:

\begin{table}[!ht]
  \centering
  \begin{tabular}{ll}
    \toprule
    String method/function & What it does \\
    \midrule
    \pythoninline!len(s)! & Length of \pythoninline!s! \\
    \pythoninline!r = s.upper()! & replaces all lowercase characters with uppercase
    and puts that in \pythoninline!r! \\
    \pythoninline!r = s.lower()! & replaces all uppercase characters with lowercase
    and puts that in \pythoninline!r! \\
    \pythoninline!r = s.capitalize()! & lower case of \pythoninline!s! with capital
    first letter and puts that in \pythoninline!r! \\
    \pythoninline!r = s.isnumeric()! & \pythoninline!r! is \pythoninline!True! if
    \pythoninline!s! contains only numbers and \pythoninline!False! otherwise\\
    \pythoninline!r = s.isalpha()! & \pythoninline!r! is \pythoninline!True! if
    \pythoninline!s! contains only alphabet characters and \pythoninline!False!
    otherwise\\
    \pythoninline!r = s.islower()! & \pythoninline!r! is \pythoninline!True! if
    \pythoninline!s! contains only lower case characters and \pythoninline!False!
    otherwise\\
    \pythoninline!r = s.isupper()! & \pythoninline!r! is \pythoninline!True! if
    \pythoninline!s! contains only upper case characters and \pythoninline!False!
    otherwise\\
    \pythoninline!r = s.replace(x, y)! & replace any occurence of \pythoninline!x! in
    \pythoninline!s! with \pythoninline!y! and put result in \pythoninline!r! \\
    \pythoninline!r = s.join([a, b, c])! &
    convert each element in the list to a string and join them with \pythoninline!s!
    \\
    \bottomrule
  \end{tabular}
  \caption{String methods and functions, where \texttt{s} is a string}
  \label{tab:str}
\end{table}

\pagebreak
\section{Exercises}
\label{sec:ex}

%\begin{warningbox}{Requirements}
%  Remember to adhere to PEP 8, PEP 257, and the boilerplate code requirement.
%\end{warningbox}

\begin{warningbox}{Requirements}
  Please be aware of the coding requirements: PEP 8, PEP 257
  and the boilerplate code.
\end{warningbox}

\begin{ex}[all\_different.py]
For this exercise you will write a function called  \texttt{check} that takes
a list as an argument. It returns \pythoninline{True} if no element is
repeated or if the list is empty. It should return \pythoninline{False}
otherwise; this means there are two or more elements in the list that are equal.

To test this file you can import it while in the Python console or you can
create a main function.

\begin{pyconcode}
>>> import all_different
>>> all_different.check([])
True
>>> all_different.check([1,2,3])
True
>>> all_different.check([7,5,1,2,4])
True
>>> all_different.check([1,1])
False
>>> all_different.check([1,2,3,4,5,1])
False
\end{pyconcode}
\end{ex}

\begin{ex}[four\_bulls.py]
This exercise depends on the previous exercise \texttt{all\_different.py}. The \texttt{check} function will be used. Please see the template files \texttt{all\_different.py} and \texttt{four\_bulls.py}.

For this exercise you must write a Python script implementing the game
called Bulls and Cows. Here's how it works.
The computer should generate a 4-digit secret number, but for this lab we will
always set the secret number to 4271. Note all the digits in the secret
number  must be different. Then the user tries to guess the computer's number
by typing four digit numbers. The computer should print a warning if the four
digits are not all different or if something other than four digits is typed.
You should use import and use the function you wrote for the previous exercise.

If the user's input is valid, the computer should print the number of digits
that were guessed correctly. If the matching digits are in their right
positions, they are called ``bulls'', if they are in the secret key but in
different positions, they are ``cows''. Example:

\begin{description}
\item[Secret number:] 4271
\item[Guess:] 1234
\item[Answer:] 1 bull and 2 cows. The bull is ``2'', the cows are ``4'' and ``1''.
\end{description}

Output should resemble the following.

\begin{verbatimcode}
Try to guess the secret number.
Please type four unique digits: 1234
1 bulls and 2 cows
Please type four unique digits: 4234
Repeated digits detected, please try again.
Please type four unique digits: 12
Four digits must be entered, please try again.
Please type four unique digits: no?
Four digits must be entered, please try again.
Please type four unique digits: 123456789
Four digits must be entered, please try again.
Please type four unique digits: 4231
3 bulls and 0 cows
Please type four unique digits: 4237
2 bulls and 1 cows
Please type four unique digits: 4271
4 bulls, you win!
\end{verbatimcode}

Hint: represent the secret key as a list instead of a single number.
\end{ex}

\begin{infobox}{Supplementary Files}
Download the template files \texttt{all\_different.py} and \texttt{four\_bulls.py}
from Canvas.
\end{infobox}

\begin{ex}[luhns.py] Luhn's algorithm
 (\url{http://en.wikipedia.org/wiki/Luhn_algorithm}) provides a quick way to
 check if a credit card is valid or not. The algorithm consists of four
 steps.
 We will use the Diners Club card number 38520000023237 as an example.

 \begin{enumerate}

\item Starting with the second to last digit (ten's column digit),
     multiply every other digit by two.

 \begin{center}
 \begin{tabular}{llllllllllllll}
3 & 5  & 2 & 0 & 0 & 0 & 0 & 0 & 2 & 3 & 2 & 3 & 7\\
 \end{tabular}
 \end{center}

\item Then double the digits.
 \begin{center}
 \begin{tabular}{llllllllllllll}
6 & 10 & 2 & 0 & 0 & 0 & 0 & 0 & 2 & 6 & 2 & 6 & 7
 \end{tabular}
 \end{center}

\item Sum all the digits of the resulting number.
 Note that for 10, you also sum its digits: $(1 + 0)$; for an 18, you would do $1 + 8$.
 \[ 6 + 8 + (1 + 0) + 2 + 0 + 0 + 0 + 0 + 0 + 2 + 6 + 2 + 6 + 7 = 40 \]

 \item If the total sum modulo by 10 is zero, then the card is valid;
     otherwise it is invalid.
 For this example, the last step is to check if $40$ modulus 10 is equal to 0, which is true.
 So the card is valid.
 \end{enumerate}

 Write a program that implements Luhn's Algorithm for validating credit
 cards. It should ask the user to enter a credit card number and tell the
 user whether it is valid or not.

 \emph{There must be a separate function called \texttt{validate}
that takes in a card number and
 validates it.}
\end{ex}

\begin{infobox}{Supplementary Files}
You will find the testing script \texttt{test\_luhns.py} (available on Canvas)
very useful for this next exercise. Read the Exercises section in lab 2 if you
don't remember how to run testing scripts.
\end{infobox}

\begin{ex}[navigate2.py] Modify \texttt{navigate.py} from lab 3 so that,
  instead of performing each action as it is entered, it stores the inputs in a
  list and runs them all at once after the ``stop'' command has been given.
\end{ex}

\section{Extra Credit Exercise}

\begin{extraex}[flatten.py]
Write a function \texttt{flatten} that takes a list containing either
lists or integers, strings or booleans and returns a list containing
nothing but the integers, strings and booleans that were inside every
sublist. There is no limit to how deeply lists will be nested.
For example:

\begin{pyconcode}
>>> import flatten
>>> flatten.flatten([])
[]
>>> flatten.flatten([1,2,3])
[1,2,3]
>>> flatten.flatten([1,2,3,[4]])
[1,2,3,4]
>>> flatten.flatten([[1,2],[3,[[[[[[[4]]]]]]]], 5])
[1,2,3,4,5]
\end{pyconcode}

The list does not have to return the elements in the same order.
You may assume there are no repeated elements.
For example:
\begin{pyconcode}
>>> flatten.flatten([[[[]]],[["string", True], 5], 7])
[True, 7, 5, "string"]
\end{pyconcode}
\end{extraex}

\begin{infobox}{Supplementary Files}
The testing script \texttt{test\_flatten.py} is available on Canvas.
\end{infobox}

\pagebreak
\section{Submitting}

You should submit your code as a tarball. It should contain all files
used in the exercises for this lab. The submitted file should be named
\begin{center}
  \texttt{cse107\_firstname\_lastname\_lab4.tar.gz}
\end{center}

\begin{center}
  \textbf{Upload your tarball to Canvas.}
\end{center}

\listexercises
\listextraexercises

\end{document}