% LAB 3: Functions
%
% CSE/IT 107: Introduction to Programming
% New Mexico Tech
%
% Prepared by Russell White and Christopher Koch
% Spring 2015

% - Functions
%   - Positional and Optional Arguments
%   - Recursion
% - Modules
%   - import statements
%   - main() boilerplate
% - Comments and PEP-8 style
% - Keywords: def
\documentclass[11pt]{cselabheader}
\fancyhead[R]{Lab 3: Functions}
\title{Lab 3: Functions}

\begin{document}

\pagenumbering{roman}
\maketitle

\hrule
\begin{quotation}
  ``If you don't think carefully, you might believe that programming is just
  typing statements in a programming language.''
\end{quotation}
\begin{flushright}
  --- W. Cunningham
\end{flushright}

\begin{quotation}
  ``Only ugly languages become popular. Python is the exception.''
\end{quotation}
\begin{flushright}
  --- Donald Knuth
\end{flushright}

\begin{quotation}
  ``need funny quote.''
\end{quotation}
\begin{flushright}
  --- Chris
\end{flushright}

\hrule

\section{Introduction}
\label{sec:intro}

\section{Conventions}
\label{sec:pep8}

pep8, some background

\section{Functions}
\label{sec:funcs}

basic functions, def

\subsection{Summary}
\label{subsec:funcs.sum}

\subsection{Exercises}
\label{subsec:funcs.ex}

\section{Modules}
\label{sec:modules}

import, boilerplate

\subsection{Summary}
\label{subsec:modules.sum}

\subsection{Exercises}
\label{subsec:modules.ex}

\section{Advanced Functions}
\label{sec:adv}

\subsection{Positional and Optional Arguments}
\label{subsec:adv.args}

\subsection{Recursion}
\label{subsec:adv.recursion}

\subsection{Summary}
\label{subsec:adv.sum}

\subsection{Exercises}
\label{subsec:adv.ex}

\pagebreak
\section{Submitting}

Files to submit:
\begin{itemize}
  \item 
\end{itemize}

You should submit your code as a tarball. It should contain all files
used in the exercises for this lab. The submitted file should be named
\begin{center}
  \texttt{cse107\_firstname\_lastname\_lab3.tar.gz}
\end{center}

\begin{center}
  \textbf{Upload your tarball to Canvas before the deadline.}
\end{center}


\end{document}
