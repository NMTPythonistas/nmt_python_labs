% LAB 5: Dictionaries and Set
% 
% CSE/IT 107: Introduction to Programming
% New Mexico Tech
% 
% Prepared by Russell White and Christopher Koch
% Spring 2015
\documentclass[11pt]{cselabheader}
\usepackage{IEEEtrantools}

%%%%%%%%%%%%%%%%%% SET TITLES %%%%%%%%%%%%%%%%%%%%%%%%%
\fancyhead[R]{Lab 5: Dictionaries and Set}
\title{Lab 5: Dictionaries and Set}

\begin{document}

\maketitle
\pagenumbering{roman}
\hrule

\begin{quotation}
``All thought is a kind of computation.''
\end{quotation}
\begin{flushright}
--- D. Hobbes
\end{flushright}

\begin{quotation}
``Vague and nebulous is the beginning of all things, but not their end.''
\end{quotation}
\begin{flushright}
--- K. Gibran
\end{flushright}

\begin{quotation}
``It [programming] is the only job I can think of where I get to be both an
engineer and artist. There's an incredible, rigorous, technical element to it,
which I like because you have to do very precise thinking. On the other hand, it
has a wildly creative side where the boundaries of imagination are the only real
limitation.''
\end{quotation}
\begin{flushright}
--- A. Hertzfeld
\end{flushright}

\hrule

\section{Introduction}

In previous labs, we have used lists and tuples to be able to enclose data
in a certain
structure and manipulate it. Lists gave us an easy way to find sums of numbers,
to store things, to sort things, and much more. Structures like this are
commonly referred to as collections: collections collect data and encapsulate
it. They give us useful methods to manipulate that data.

A list in Python is an ordered container of any elements you want indexed by
whole numbers starting at 0. Lists are mutable: this means you can add elements,
change elements, and remove elements from a list. Meanwhile, tuples are immuatable
and cannot be changed once they are created.
In this lab, we will introduce you to two other collections: Sets and dictionaries.

\pagebreak

\tableofcontents

\pagebreak

\pagenumbering{arabic}
\section{Sets}
Sets are a lot like lists, but their properties are a bit different.
A set is just like a list, but \emph{no element can appear twice} and it is
\emph{unordered}. Additionally, they cannot contain mutable elements. Thus,
a set cannot contain a list. Sets themselves are mutable.

A set is made using curly braces or from a list:
\begin{pyconcode}
>>> a = {5, 5, 4, 3, 2} # duplicates ignored
>>> print(a)
{2, 3, 4, 5}
>>> b = set([5, 5, 4, 3])
>>> print(b)
{3, 4, 5}
\end{pyconcode}

Sets are heterogeneous -- their elements do not need to be of the same type:
\begin{pyconcode}
>>> a = {5, 4, 'a'}
>>> print(a)
{'a', 4, 5}
\end{pyconcode}

Because sets are inherently unordered, they cannot be indexed. That means you
cannot use the \pythoninline![]! operator on sets:
\begin{pyconcode}
>>> a = {5, 4, 3, 2}
>>> a[1]
Traceback (most recent call last):
  File "<stdin>", line 1, in <module>
TypeError: 'set' object does not support indexing
\end{pyconcode}


The \pythoninline!|! operator can be used to join sets together. This results
in a new set composed of the elements of the two composite sets. This is called
the \emph{union} of the two sets. Since the result is also a set, any duplicate
values will be only included once.
\begin{pyconcode}
>>> a = {1, 2, 4, 8, 16}
>>> b = {1, 3, 9, 27}
>>> a | b
{1, 2, 3, 4, 8, 9, 16, 27}
\end{pyconcode}

The \pythoninline!&! operator can be used to find the intersection of two sets.
This is the set containing all elements that are in both sets.
\begin{pyconcode}
>>> a = {2, 4, 6, 8, 10, 12}
>>> b = {3, 6, 9, 12, 15, 18}
>>> a & b
{12, 6}
\end{pyconcode}

The \pythoninline!-! operator can be used to find the difference of two sets.
This is the first set with all of the elements of the second set removed.
\begin{pyconcode}
>>> a = {2, 4, 6, 8, 10, 12}
>>> b = {3, 6, 9, 12, 15, 18}
>>> a - b
{8, 2, 10, 4}
\end{pyconcode}

The \pythoninline!^! operator can be used to find the symmetric difference
of two sets. This is the set containing all elements in one of the original
sets, but not in both.
\begin{pyconcode}
>>> a = {'one', 'eight'}
>>> b = {'two', 'four', 'six', 'eight'}
>>> a ^ b
{'one', 'six', 'two', 'four'}
\end{pyconcode}

A list can be converted into a set using the \pythoninline!set()! function:
\begin{pyconcode}
>>> a = [1, 5, 2, 'hello', 2.3, 5, 2]
>>> set(a) # duplicate removed!
{1, 2, 'hello', 2.3, 5}
\end{pyconcode}

Similarly, a set can be converted into a list using the \pythoninline!list()!
function:
\begin{pyconcode}
>>> a = {1, 2, 2, 5, 'asdf', 3, 2}
>>> list(a)
[1, 2, 3, 5, 'asdf']
\end{pyconcode}

You may also iterate over a set and add elements using the
\pythoninline!.add()! method:
\begin{pyconcode}
>>> companions_nine = {'rose', 'jack'}
>>> companions_ten = {'rose', 'mickey', 'jack', 'donna', 'martha', 'wilf'}
>>> # iterating over elements
... for companion in companions_nine:
...     print(companion)
...
rose
jack
>>> # adding an element to set
... companions_ten.add('sarah jane')
>>> companions_ten
{'donna', 'sarah jane', 'wilf', 'rose', 'martha', 'mickey', 'jack'}
\end{pyconcode}

\pagebreak
\subsection{Summary}

\begin{itemize}
  \item Sets can be made using curly braces or using the \pythoninline!set()!
    function given another sequence type:
    \begin{python3code}
food = {'burgers', 'carne adovada', 'burritos'}
food = set(['burgers', 'carne adovada', 'burritos'])
food = set(('burgers', 'carne adovada', 'burritos'))
    \end{python3code}

  \item No element can appear twice.

  \item Can only contain immutable elements (for example, lists are not allowed
    as elements, but tuples are as long as the tuple only contains immutable
    elements).

  \item Given two sets \pythoninline!a! and \pythoninline!b!:
    \begin{tabular}{ll}
      \pythoninline!a | b! & union of \pythoninline!a! and \pythoninline!b! \\
      \pythoninline!a & b! & intersection of \pythoninline!a! and \pythoninline!b! \\
      \pythoninline!a - b! & difference of \pythoninline!a! and \pythoninline!b! \\
      \pythoninline!a ^ b! & symmetric difference of \pythoninline!a! and \pythoninline!b! \\
    \end{tabular}

  \item Python docs:

    \url{https://docs.python.org/3.3/tutorial/datastructures.html#sets}
\end{itemize}

\pagebreak
\section{Dictionaries}
Dictionaries are similar to lists, but with a different index system.
The elements of a dictionary are stored using a key-value system. A key
is used similarly to the indices of a list, but it must be of any immutable
type. A value is simply the element associated with that key and can be
anything. This means that dictionaries are useful for storing elements best
identified by a description or name rather than a strict order.

\begin{itemize}
  \item A dictionary is made using curly braces, followed by listing the
    key-value pairs that make up the dictionary. Keys and values are separated
    by colons, while key-value pairs are separated by commas.
  \item Values can be accessed, changed, or added by using their keys just as
    you would use the indices of a list.
\end{itemize}
\begin{pyconcode}
>>> food = {'breakfast': 'burrito', 'lunch': 'burger', 'dinner': 'tacos'}
>>> print(food)
{'dinner': 'tacos', 'lunch': 'burger', 'breakfast': 'burrito'}
>>> food['breakfast'] = 'toast'
>>> print(food)
{'dinner': 'tacos', 'lunch': 'burger', 'breakfast': 'toast'}
>>> food['brunch'] = 'biscuits'
>>> print(food)
{'dinner': 'tacos', 'brunch': 'biscuits', 'lunch': 'burger', 'breakfast': 'toast'}
\end{pyconcode}

\begin{itemize}
  \item Like sets, the elements of a dictionary are unordered.

  \item A list of the keys of a dictionary can be accessed using the
    \pythoninline!.keys()! function.
\end{itemize}

\begin{pyconcode}
>>> a = {1: 5, 'cake': 'lots', 'color': 'green'}
>>> print(a)
{1: 5, 'cake': 'lots', 'color': 'green'}
>>> print(a.keys())
dict_keys([1, 'cake', 'color'])
\end{pyconcode}

\begin{itemize}
  \item Only immutable elements can be used as keys.
  \item (Actually, only \emph{hashable} elements can be used as keys; however,
    this usually means immutable. The error message will tell you that a type is
    \emph{unhashable} when you used a mutable element as a key or set element.)
  \item Since tuples are immutable, they can be used as keys for dictionaries.
    However, the tuple must not contain anything mutable, like a list,
    dictionary, or set.
\end{itemize}

\begin{pyconcode}
>>> a = {}
>>> a[('a', 'b')] = 5
>>> print(a)
{('a', 'b'): 5}
>>> a[('a', 'c')] = 6
>>> print(a)
{('a', 'c'): 6, ('a', 'b'): 5}
>>> a['a', []] = 7
Traceback (most recent call last):
  File "<stdin>", line 1, in <module>
TypeError: unhashable type: 'list'
\end{pyconcode}

\begin{itemize}
  \item You can use the \pythoninline!in! keyword to test whether an item is a
    key in the respective dictionary.
  \item You can use a for loop to iterate over the \emph{keys} of a dictionary.
  \item \pythoninline!.items()! returns a list of tuples where the first element
    is the key of a dictionary item and the second element is the corresponding
    value. This can be used in a for loop as well with the unpacking syntax.
\end{itemize}

\begin{pyconcode}
>>> states = {'NM' : 'New Mexico', 'TX' : 'Texas', 'KS' : 'Kansas'}
>>> 'NM' in states
True
>>> for state_short in states:
...     print(state_short, states[state_short])
...
KS Kansas
TX Texas
NM New Mexico
>>> states.items()
dict_items([('KS', 'Kansas'), ('TX', 'Texas'), ('NM', 'New Mexico')])
>>> for state_short, state_name in states.items():
...     print(state_short, state_name)
...
KS Kansas
TX Texas
NM New Mexico
\end{pyconcode}

Dictionaries are good for a load of things. Here's an example:

\begin{python3code}
states = {'NM' : 'New Mexico', 'TX' : 'Texas', 'KS' : 'Kansas'}
capitals = { # multiline dictionaries are easier to read
  'NM' : 'Santa Fe',
  'TX' : 'Austin',
  'KS' : 'Kansas City',
}

state = input('Enter a state >>> ')
if state in states:
    print('You selected {}. The state capital is {}.'.format(states[state], 
        capitals[state]))
else:
    print('The state you selected is not known to this program.')
\end{python3code}

\pagebreak
\section{Collection Similarities}


\section{Collections Summary}

% this could be something better than a table
\begin{table}[!ht]
  \centering
  \begin{tabular}{p{1.6cm}lp{1.6cm}p{3.5cm}lp{4.5cm}}
    \toprule
    Data Structure & Mutable & Mutable elements & Indexing & Ordered
    & Other
    properties\\
    \midrule
    List \lstinline![]! & yes & yes & by whole numbers & yes & can contain elements more than once\\
    Sets \lstinline!{}! & yes & no & not indexable & no & no element can appear more than once\\
    Tuples \lstinline!()! & no & yes & by whole numbers & yes & can contain elements more than
    once\\
    Dictionary \lstinline!{}! & yes & yes & by anything ``hashable'' (immutable collections or
    basic types) & no & \\
    \bottomrule
  \end{tabular}
  \caption{Summary of Data Structures in Python}
  \label{tab:sum}
\end{table}

For syntax, please refer to the Python Tutorial or Chapters 7 and 9 in your
book.
\begin{center}
  \url{https://docs.python.org/3.3/tutorial/datastructures.html}
\end{center}
Be sure to use the Python 3 version of the Python Tutorial.



%\subsection{List Syntax}
%
%\begin{lstlisting}[style=ipython]
%>>> semester_gpas = [4.0, 3.5, 2.0, 1.5, 4.0, 3.7]
%>>> 4.0 in semester_gpas
%True
%>>> 1.0 not in semester_gpas
%True
%>>> 2.0 not in semester_gpas
%False
%>>> mult_gpas = 1
%>>> for gpa in semester_gpas:
%      mult_gpas += gpa
%>>> print(mult_gpas)
%621.6
%>>> semester_gpas[2]
%2.0
%>>> semester_gpas[:2]
%[4.0, 3.5]
%>>> semester_gpas.append(3.4)
%>>> semester_gpas
%[4.0, 3.5, 2.0, 1.5, 4.0, 3.7, 3.4]
%\end{lstlisting}
%
%Refer to lab 3 for more list functions and the slicing notation.
%
%\subsection{Set Syntax}
%
%\begin{lstlisting}[style=ipython]
%>>> companions_nine = {'rose', 'jack'}
%>>> companions_ten = {'rose', 'mickey', 'jack', 'donna', 'martha', 'wilf'}
%# intersection
%>>> companions_of_both = companions_nine & companions_ten
%>>> print(companions_of_both)
%{'jack', 'rose'}
%# union
%>>> print(companions_nine | companions_ten)
%{'donna', 'wilf', 'rose', 'martha', 'mickey', 'jack'}
%# testing membership
%>>> 'rose' in companions_nine
%True
%# iterating over elements
%>>> for companion in companions_nine:
%      print(companion)
%
%rose
%jack
%# adding an element to set
%>>> companions_ten.add('sarah jane')
%>>> print(companions_ten)
%{'donna', 'sarah jane', 'wilf', 'rose', 'martha', 'mickey', 'jack'}
%\end{lstlisting}
%
%\subsection{Tuple Syntax}
%
%\begin{lstlisting}[style=ipython]
%>>> ordered_pairs = [(1,2), (2,1), (3,3), (4,5), (5,4)] # list of tuples
%\end{lstlisting}
%
%\subsection{Dictionary Syntax}
%
%\begin{lstlisting}[style=ipython]
%ages = {'chris': 20, 'tyler': 22, 'randomperson': 34} # dictionary
%\end{lstlisting}


\pagebreak
\section{Stacks}
\label{sec:stacks}

Stacks are an important data structure in the world of computer
science. They are at the heart of every operating system and used in many, many
pieces of software.

For a stack, imagine a stack of plates. You can only add plates to the top and
remove plates from the top. We call this a ``Last-In-First-Out'' data structure:
If you add three plates to your stack, the last one you added will be the first
one you remove.

Similar to that, in Python you can say that a stack
is a list where you can only add elements to the end or remove elements from the
end. In Python, this is accomplished using the
\lstinline!.pop()! and \lstinline!.append(element)! methods on lists. When given
no arguments, the pop method will remove the last element of the list and return
it. The append method will add an element to the end of the list.

When we ask you to use a stack in Python, you should use a list and only use
these two methods and the array index \lstinline![-1]! to inspect the last
element of the list. For example:

\begin{pyconcode}
>>> somelist = [1, 2, 3]
>>> a = somelist.pop()
>>> print(a)
3
>>> print(somelist)
[1, 2]
>>> somelist.append(4)
>>> print(somelist)
[1, 2, 4]
>>> somelist[-1]
4
\end{pyconcode}

Interacting with the elements of the list in any other way violates the
idea of the list being a stack.

\pagebreak

\section{Exercises}
\label{sec:ex}

\begin{warningbox}{Requirements}
  Remember PEP 8, PEP 257, and the boilerplate code requirement.
\end{warningbox}

\begin{ex}[days.py] Write a function that takes four arguments -- day, month, and
    year as numbers, and weekday as MTWRFSU -- and converts this date to a
    human-readable string. Have the program be called with user-specified
    input. For example:
    \begin{lstlisting}[style=bash]
Enter day: 28
Enter month: 9
Enter year: 2014
Enter weekday: U
Date is: Sunday, September 28, 2014
    \end{lstlisting}

    Use dictionaries to convert weekdays to their long name and months to their
    long name.
\end{ex}

\begin{ex}[addresses.py]
  Write a small address book program that keeps track of addresses, phone
  numbers, and names. Your program should contain multiple functions and must
  use dictionaries. The interface would look something like this:

\begin{verbatimcode}

\end{verbatimcode}
\end{ex}

%  \item[zombies.py] Zombie lab from 113. Use struct like dict and array like
%    list.

%  \item[words.py] Write program that takes a file and
%    \begin{enumerate}
%      \item finds how many unique words there are
%      \item finds the unique words and writes them to
%        \texttt{originalfile\_unique.txt}
%      \item calculate the percentage of unique words
%      \item MORE
%    \end{enumerate}
%
  %\item[data.py] Some data analysis exercise. Chris will fill in. Use tuples,
  %  use matplotlib. Use set to remove duplicates in data.
  % exercise will be made available in the second week. no time to write about
  % matplotlib.

\begin{ex}[rpn\_calculator.py] Your task is to write a reverse Polish notation
    calculator. In reverse Polish notation (also HP
    calculator notation), mathematical expressions are written with the operator
    following both operands. For example, $3 + 4$ becomes $3~4~+$.

    To write $3 + (4 * 2)$, we would have to write $4~2~*~3~+$ in RPN. The
    expressions are evaluated from left to right.

    A longer example of an expression is this: 
    \[ 5~1~2~/~4~*~+~3~- \]
    which translates to
    \[ 5 + ( (1 / 2) * 4 ) - 3 \]

    If you were to try to ``parse'' the RPN expression from left to right, you
    would probably ``remember'' values until you hit an operator, at which point
    you take the last two values and use the operator on them. In the example
    expression above, you would store 5, then 1, then 2, then see the division
    operator (/) and take the last two values you stored (1 and 2) to do the
    division. Then, you would store the result of that (0.5) and encounter
    4, which you store. When you encounter the multiplication sign (*), you
    would take the last two values stored and do the operation ($4 * 0.5$)
    and store that. 

    Writing this algorithm for evaluating RPN in pseudo code, we get:

\begin{enumerate}
  \item Read next value in expression.
  \item If number, store.
  \item If operator:
    \begin{enumerate}
      \item Remove last two numbers stored.
      \item Do operation with these last two numbers.
      \item Store the result of the operation as last number.
    \end{enumerate}
\end{enumerate}

    If you keep repeating this algorithm, you will eventually just end up with
    one number stored unless the RPN expression was invalid.

    Your task is to write an RPN calculator which asks the user for an RPN
    expression and prints the result of that expression. You \emph{must} use a
    stack (see Section~\ref{sec:stacks}). The RPN algorithm has to be in a
    separate function (not main). You need to support the four basic operators
    (+, -, *, and /).

    Please see the example file and output below for expressions you can test
    with.

  \end{ex}

\pagebreak
\section{Submitting}

You should submit your code as a tarball. It should contain all files
used in the exercises for this lab. The submitted file should be named
\begin{center}
  \texttt{cse107\_firstname\_lastname\_lab5.tar.gz}
\end{center}

\begin{center}
  \textbf{Upload your tarball to Canvas.}
\end{center}

\listoftheorems

\end{document}
