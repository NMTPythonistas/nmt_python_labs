% LAB 7: Programming Competition
% 
% CSE/IT 107: Intro to Programming in Python
% New Mexico Tech
% 
% Prepared by Christopher Koch
% Fall 2014
\documentclass[11pt]{cselabheader}
% formatting the problem
\newcommand{\cop}[3]{\textbf{#1}~(#2 pts, filename: \texttt{#3})\quad}
\usepackage{IEEEtrantools} 
\theoremstyle{plain}
\newtheorem{thm}{Theorem}
% drawings in latex
\usepackage{tikz}

\fancyhead[R]{Lab 7: Programming Competition}
\title{Lab 7: Programming Competition}

\begin{document}

\maketitle

\hrule

\begin{quotation}
  ``First, solve the problem. Then, write the code.''
\end{quotation}
\begin{flushright}
  --- John Johnson
\end{flushright}

\begin{quotation} ``Beware of bugs in the above code; I have only proved it
  correct, not tried it.''
\end{quotation}
\begin{flushright}
  --- Donald Knuth
%\\\textit{Notes on the van Emde Boas construction of priority deques}
%% From corresponde with Peter van Emde Boas about Priority Deques
%% http://www-cs-faculty.stanford.edu/~uno/faq.html
\end{flushright}

\begin{quotation}
  ``Openness always deserves recognition.'' (``Offenheit verdient immer
  Anerkennung.'')
\end{quotation}
\begin{flushright}
  --- Otto von Bismarck
% \\\textit{Speeches in the second chamber of Prussian state diet
% (parliament). November 24th, 1849.}

%% Reden in der preußischen Zweiten Kammer, 24. November 1849
%% http://de.wikiquote.org/wiki/Otto_von_Bismarck
%% Translation by Chris
\end{flushright}

\hrule

\pagebreak

\section{Introduction}
This lab is solely a problem solving lab. There is nothing to turn in.

\section{Programming Competition}
Please form groups of up to three with others in your lab. You will
enter into the competition together and submit the same files.

%To submit your answers, find the scoreboard at 
%\begin{center}
%	\vspace{-5mm}
%	\url{http://niedersfeld.cs.nmt.edu/scoreboard/}
%	\vspace{-5mm}
%\end{center}
%Upon submission of your answer, someone will come and check the code to make
%sure you actually found the answer using code and did not use the internet.

To submit your answers, please notify a TA. The whiteboard will serve as the
scoreboard for your lab. The TA will want to see the answer to the ``scoreboard
submission.'' If the TAs are grading another team's submission,  move on to a
new problem while waiting for them to be free.

\textbf{Do not share the problem descriptions or the code with the other lab
section.} Everyone needs to have an equally fair chance of doing well in this
competition.

You may not use the internet to find answers to these problems. Think! If the
concepts are unclear, please get help from a TA.

\subsection{Extra Credit}
This lab will not count for its own grade. Your standing in the competition 
will count for extra credit on your lab final.
The winners of your lab section will get 10\%, the second place will get
9\%, and so on as extra credit on their lab final. 

You have to gain at least 20 points in the competition in order to qualify
for the extra credit.

\subsection{Requirements}
Even though this lab ``only'' counts for extra credit, you have to stay the lab
and try your best. \textbf{You will LOSE 10\% on your lab final if you do not
stay and try your best.}

You may choose to do any of the following problems. However, there is no credit
for incomplete or incorrect submissions.

%You may use Google to help find solutions to the problems. However,
%there are two conditions to this: (1) you \textbf{must}
%cite the sources you use in your comments (URLs, etc) and (2) you \textbf{must}
%understand the content you use from these sources. \textbf{Your grader will ask
%you questions to make sure you understand.}

\pagebreak
\subsection{Problems}

\begin{enumerate}[\text{Problem} {[}1{]}~]
  \item \cop{Multiples of 3 and 5}{5}{multiples.py}
    % project euler 1

    If we list all the integers (natural numbers) below 10 that are
    multiples of 3 or 5, we get 3, 5, 6, and 9. The sum of these multiples
    is 23.

    Write a program to find the sum of all the multiples of 3 or 5 below
    1000.

    \textit{Scoreboard submission:} The sum of all multiples of 3 or 5
    below 1000.

  \item \cop{Special Pythagorean Triplet}{10}{pythagoras.py}
    % project euler 9

    A Pythagorean triplet is a set of three integers (natural numbers), $a
    < b < c$ such that $a^2 + b^2 = c^2$.

    For example, $a = 3, b = 4, c = 5$ fits: $3^2 + 4^2 = 9 + 16 = 25 =
    5^2$.

    There exists exactly one Pythagorean triplet for which $a + b + c =
    1000$.

    \textit{Scoreboard submission:} Find the product $abc$ of that special
    Pythagorean triplet.

  \item \cop{Reducing Fractions}{10}{fractions.py}
    % Source: ACM Freshman Coding Competition, Fall 2013 (Russell White)

    Write a program that reads both the numerator and denominator of
    a fraction and displays the reduced fraction. For simplicity, you
    may assume that the numbers entered are less than 10,000. 

    For example:

    \begin{verbatimcode}
Enter numerator: 4
Enter denominator: 30
Result: 2/15
    \end{verbatimcode}

    \begin{verbatimcode}
Enter numerator: 25
Enter denominator: 5
Result: 5/1
    \end{verbatimcode}

    \textit{Hints}: See Appendix~\ref{sec:gcd} to find out what a GCD is 
    and what Euclid's algorithm is. 

    \textit{Scoreboard submission:} Contact a TA for test cases.

  \item \cop{Crossing Hands}{20}{hands.py}
    % source: http://programmingpraxis.com/2014/02/25/crossing-hands/

    The hands of an analog clock occassionally cross as they revolve around
    the dial. Write a program that determines the times at which times the 
    hour and minute hands cross. Assume that
    the hands move continuously (that is, the hands do not jump between
    numbers -- the times at which they cross may not necessarily be
    integer values).

    \textit{Scoreboard submission} Add up the hours, minutes, and seconds
    of all of the times that the hour and minute hands cross
    and submit it in the format \texttt{H-M-S}, 
    for example \texttt{30-167-421}.


  \item \cop{Anagram Distance}{20}{anadist.py}
    
    The anagram distance between two words is how many letters must be added
    or removed from one of the words in order to make it an anagram of the
    other. For example, ``dog'' and ``godot'' have an anagram distance of $2$
    while ``sandwich'' and ``witch'' have an anagram distance of $5$.

    You can assume that input will be purely lowercase letters and will be
    no longer than 20 characters.

    \begin{verbatimcode}
Enter first word: sandwich
Enter second word: witch
Result: 5
    \end{verbatimcode}

    \begin{verbatimcode}
Enter first word: interwebs
Enter second word: spiderman
Result: 8
    \end{verbatimcode}

    \textit{Scoreboard submission} Contact a TA for test cases.


  \item \cop{Contiguous Regions}{25}{regions.py}
    
    Write a program that takes in a square region and determines whether two
    points are connected. The region will consist of three characters: \texttt{.},
    \texttt{x}, and \texttt{s}. There will be exactly two \texttt{s}'s while there
    can be any number of \texttt{.}'s and \texttt{x}'s. Your program will output
    either \texttt{connected} or \texttt{not connected}, depending on if the two
    \texttt{s} can be travelled between by only moving across \texttt{.}'s and only
    moving in cardinal directions.

    \begin{verbatimcode}
..s.
.xxx
.x..
...s
connected
    \end{verbatimcode}

    \begin{verbatimcode}
....x
.s.x.
..x..
.x.s.
x....
not connected
    \end{verbatimcode}

    \textit{Scoreboard submission} Contact a TA for test cases.

    

  \item \cop{Finding a Path}{25}{path.py}
    %Source: ENMU Codecomp, Spring 2014 (Tyler Cecil)
    
    Write a program that finds the shortest valid path between two nodes in a graph.
    Each of the lines of input will be the name of a node (a single, upper-case
    character), followed by a list of the nodes connected to that node. The final
    line of input will a single upper-case letter, designating the destination node.
    You must output a sequence of letters that represents the shortest path to the
    destination. Assume you start at the first given node. You can assume a path will
    always exist. There will be less than 10 nodes given. If multiple valid paths
    exist, any will be accepted.
    
    For example:\\
    \begin{minipage}{.33\linewidth}
      \begin{minipage}{.8\linewidth}
        \begin{lstlisting}[style=bash]
A B D
B C D
C

A B C
        \end{lstlisting}
      \end{minipage}
    \end{minipage}
    \begin{minipage}{.33\linewidth}
      \begin{minipage}{.8\linewidth}
        \begin{lstlisting}[style=bash]
A B C D
B F
D B C
F E
E

A B F E
        \end{lstlisting}
      \end{minipage}
    \end{minipage}
    \begin{minipage}{.33\linewidth}
      \begin{minipage}{.8\linewidth}
        \begin{lstlisting}[style=bash]
A B C D E
B C D E F
C D E F
D E F
E F
F

A B F
        \end{lstlisting}
      \end{minipage}
    \end{minipage}

    \textit{Scoreboard submission:} Contact a TA for test cases.

  \item \cop{Block Areas}{35}{blocks.py}
    % Source: ACM Freshman Coding Competition, Fall 2013 (Russell White)
    % sample solution: http://www.mathopenref.com/coordpolygonarea2.html

    Write a program to take in a series of coordinate points and finds the 
    enclosed area. The points will all be non-negative integers not 
    exceeding 100. All angles in the shape will be 90 degrees. You can 
    assume that no shape will be defined by more than 20 points. 

    For example:
    \begin{figure}[!ht]
      \centering
      \begin{tikzpicture}[scale=0.7]
        \node at (2,1) {};% {(2,1)};
        \node at (4,1) {};% {(4,1)};
        \node at (4,3) {};% {(4,3)};
        \node at (3,3) {};% {(3,3)};
        \node at (3,5) {};% {(3,5)};
        \node at (1,5) {};% {(1,5)};
        \node at (1,4) {};% {(1,4)};
        \node at (2,4) {};% {(2,4)};

        \draw (1,5) -- (3,5);
        \draw (1,5) -- (1,4);
        \draw (1,4) -- (2,4);
        \draw (2,4) -- (2,1);
        \draw (2,1) -- (4,1);
        \draw (4,1) -- (4,3);
        \draw (4,3) -- (3,3);
        \draw (3,3) -- (3,5);
      \end{tikzpicture}
      \caption{Example shape}
    \end{figure}

    \emph{Hint:} The order that the points are input shouldn't matter
    for your algorithm.

    \begin{verbatimcode}
Enter number of points: 8
Enter point: 2 1
Enter point: 4 1
Enter point: 4 3
Enter point: 3 3
Enter point: 3 5
Enter point: 1 5
Enter point: 1 4
Enter point: 2 4

Area: 7
    \end{verbatimcode}

    \textit{Scoreboard submission:} Find the area of the polygon described
    by the following 14 coordinates.
    \[
      (0, 0)~(6, 0)~(6, 1)~(2, 1)~(2, 2)~(7, 2)~(7, 1)
    \]
    \[
      (9, 1)~(9, 4)~(2, 4)~(2, 3)~(1, 3)~(1, 2)~(0, 2)
    \]

    \begin{figure}[!ht]
      \centering
      \begin{tikzpicture}[scale=0.7]
        \node at (0,0) {};% {(2,1)};
        \node at (6,0) {};% {(4,1)};
        \node at (6,1) {};% {(4,3)};
        \node at (2,1) {};% {(3,3)};
        \node at (2,2) {};% {(3,5)};
        \node at (7,2) {};% {(1,5)};
        \node at (7,1) {};% {(1,4)};
        \node at (9,1) {};% {(2,4)};
        \node at (9,4) {};
        \node at (2,4) {};
        \node at (2,3) {};
        \node at (1,3) {};
        \node at (1,2) {};
        \node at (0,2) {};

        \draw (0,0) -- (6,0);
        \draw (6,0) -- (6,1);
        \draw (6,1) -- (2,1);
        \draw (2,1) -- (2,2);
        \draw (2,2) -- (7,2);
        \draw (7,2) -- (7,1);
        \draw (7,1) -- (9,1);
        \draw (9,1) -- (9,4);
        \draw (9,4) -- (2,4);
        \draw (2,4) -- (2,3);
        \draw (2,3) -- (1,3);
        \draw (1,3) -- (1,2);
        \draw (1,2) -- (0,2);
        \draw (0,2) -- (0,0);
      \end{tikzpicture}
      \caption{Scoreboard submission shape}
    \end{figure}

  \item \cop{Nontransitive Dice}{50}{dice.py}
    %Source: ACM Freshman Coding Competition, Fall 2013 (Rob Kelly)

    A set of dice is called nontransitive if it contains three dice,
    A, B, and C, with the property that A rolls higher than B more
    than half the time, and B rolls higher than C more than half
    the time, but it's not true that A rolls higher than C more than
    half the time. Intuitively, the existence of such a set might not
    seem possible - If $A > B > C$, how could it be true that $C > A$?
    But consider the following set of dice:

    \begin{itemize}
      \item Die A has sides 3, 3, 3, 3, 3, and 6
      \item Die B has sides 2, 2, 2, 5, 5, and 5
      \item Die C has sides 1, 4, 4, 4, 4, and 4
    \end{itemize}
    By checking all possible die-rolls, we can see that A will roll
    higher than B 21 out of 36 times, and likewise B will roll higher
    than C 21 out of 36 times, but C will roll higher than A 25 out
    of 36 times. Neat, right?

    Write a program that reads an arbitrary number of six-sided dice
    configurations followed by a single 0, and displays
    ``Nontransitive'' if there exists a subset of at least three
    nontransitive dice in the set, or ``Transitive'' if no such
    nontransitive subset exists. One die will be input per line,
    with the numbers of each side seperated by spaces. All side
    numbers will be between 0 and 999. The order of the input dice
    does not matter.

    \begin{minipage}{.5\linewidth}
      \begin{minipage}{.9\linewidth}
        \begin{lstlisting}[style=bash]
5 5 5 1 1 1
3 3 3 3 3 3
4 4 4 4 0 0
6 6 2 2 2 2
8 6 7 5 3 0
0

Nontransitive
        \end{lstlisting}
      \end{minipage}
    \end{minipage}
    \begin{minipage}{.5\linewidth}
      \begin{minipage}{.9\linewidth}
        \begin{lstlisting}[style=bash]
1 2 3 4 5 6
7 8 9 10 11 12
10 20 30 40 50 60
100 200 300 400 500 600
0

Transitive
        \end{lstlisting}
      \end{minipage}
    \end{minipage}

    \textit{Scoreboard submission:} Contact a TA for test cases.
\end{enumerate}
\pagebreak

\begin{appendices}
  \section{Reducing Fractions -- GCD and Euler's Algorithm}
  \label{sec:gcd}

  Reducing a fraction involves dividing the numerator and denominator by
  the same integers until there is nothing that divides both numerator and
  denominator anymore.

  For example, $\frac{60}{6} = \frac{30}{3} = \frac{10}{1} = 10$. First, I
  divided both numerator and denominator by $2$, and then by $3$. 

  This is where the concept of a GCD comes in. GCD stands for ``greatest
  common divisor'' -- given two numbers such as $60$ and $6$, the GCD is the
  greatest whole number that both numbers are divisible by. Since both $60$ 
  and $6$ are divisible by $6$ and since there is no greater integer than $6$
  that one can divide them both by, the $\gcd(60, 6) = 6$. For another
  example, $\gcd(40, 6) = 2$.

  If you think about it more, it becomes clear that one can reduce a fraction
  by dividing both the numerator and denominator by the GCD of both. In
  essence, given a fraction $\frac{a}{b}$, the reduced fraction would be:
  \[
    \frac{a}{b} = \frac{~\frac{a}{\gcd(a, b)}~}{~\frac{b}{\gcd(a, b)}}
  \]

  How do we find the GCD though? Here, a neat algorithm called Euclid's
  algorithm comes in. Euclid's algorithm takes advantage of the fact that
  when an integer $a$ is divided by another integer $b$, we get a quotient
  $q$ and a remainder $r$. This is called the Division Algorithm (see
  Theorem~\ref{thm:div}). 

  \begin{thm}[Division Algorithm] \label{thm:div}
    Given integers $a$ and $b$, where $b > 0$, there exist unique integers
    $q$ (the quotient) and $r$ (the remainder) such that 
    \[ a = bq + r\text{, where }0 \leq r < b. \]
  \end{thm}

  For example, when $5$ is divided by $2$, we get a quotient of $2$ and a
  remainder of $1$, since $5 = 2 \times 2 + 1$. You have already taken 
  advantage of this algorithm when you validated credit card numbers. 

  In Python, we can find the quotient by doing integer division and the remainder
  using the modulus operator \%.

  \subsection{Euclid's Algorithm}

  Euclid's algorithm finds the GCD by repeatedly using the division
  algorithm. Given integers $a$ and $b$, where $a > b$, that we want to find 
  the GCD of, we
  find the quotient $q$ and $r$. If the remainder is $0$, $b$ is the GCD. If
  the remainder is not $0$, we repeat the same steps, but we let $a = b$ and
  $b = r$.

  In essence, it goes:
  \begin{IEEEeqnarray*}{RCLCL}
    a &=& bq + r &\text{ where }&0 \leq r < b \\
    b &=& rq_2 + r_2 &\text{ where }&0 \leq r_2 < r \\
    r &=& r_2q_3 + r_3 &\text{ where }&0 \leq r_3 < r_2
  \end{IEEEeqnarray*}
  So we are saying that $\gcd(a, b) = \gcd(b, r) = \gcd(r, r_2) = \cdots =
  \gcd(r_n, 0) = r_n$. Notice that the quotient is not used at all, only the
  given integers $a, b$ and the remainder $r$ matter.

  Notice that $r_n < \cdots < r_3 < r_2 < r < b$.

  It seems natural to write this with recursion. In pseudocode, with recursion:

  \begin{verbatimcode}
euclids(a, b) 
  if(a < b) 
    return euclids(b, a)

  r = a modulus b
  if(r == 0)
    return b
  else
    return euclids(b, r)
  \end{verbatimcode}
  % TODO use algorithmicx pseudocode typesetting

\end{appendices}
\end{document}
