% LAB 8: Advanced Functions
% 
% CSE/IT 107: Introduction to Programming
% New Mexico Tech
% 
% Prepared by Russell White and Christopher Koch
% Fall 2014
\documentclass[11pt]{cselabheader}
\usepackage{IEEEtrantools}

%%%%%%%%%%%%%%%%%% SET TITLES %%%%%%%%%%%%%%%%%%%%%%%%%
\fancyhead[R]{Lab 8: Advanced Functions}
\title{Lab 8: Advanced Functions}

\begin{document}

\maketitle

\hrule
\begin{quotation}
``I think there is a world market for maybe five computers.''
\end{quotation}
\begin{flushright}
--- Thomas J. Watson (Founder and Chairman, IBM)
\end{flushright}

\begin{quotation}
``Judge a man by his questions, rather than his answers.''
\end{quotation}
\begin{flushright}
--- Voltaire
\end{flushright}

\begin{quotation}
``The limits of your language are the limits of your world.''
\end{quotation}
\begin{flushright}
--- Ludwig Wittgenstein
\end{flushright}

\hrule

\section{Introduction}
\emph{Congratulations!} In your last three weeks of lab, you have reached the
basic knowledge required to work with Python in a useful way. You have learned
all the tools required to write basic programs in Python and you have applied
them successfully in the programming competition.

This week, we will teach you a bit more about functions and give you a small
window to a different kind of programming: functional programming. Here, the
word ``functional'' is not to mean ``practical'' (although it is very
practical!), but that it has to do with a lot of functions.

\pagebreak
\section{Advanced Functions}
\label{sec:advfun}

\subsection{Keyword Arguments}
\label{subsec:adv.args}

Many of the built-in functions we have been using, such as \pythoninline{input}
and \pythoninline{print}, accept a variable number of arguments. We can do the
same thing with our own functions by specifying default values for our
arguments:

\begin{pyconcode}
>>> def celebrate(times=1):
...     i = 1
...     while i <= times:
...         print("Yay!")
...         i += 1
...
>>> celebrate()
Yay!
>>> celebrate(5)
Yay!
Yay!
Yay!
Yay!
Yay!
\end{pyconcode}

Here, we have a function called \pythoninline!celebrate! which takes the
parameter variable \pythoninline!times!. If that parameter is left unspecified
in a function call, it will take the \emph{default value} of 1.

Note what happens for each call of this function. The first time it only prints
\pythoninline{"Yay!"} once, since the default argument is \pythoninline{1},
while the second call overwrites the default value and prints it five times.

While you are allowed to have a function that has some arguments with default
values and some without, you must always put those with default values after
those without any, so that Python will know which arguments go into which values
if you do not include all of them when calling the function. This is because
arguments from a function call are usually matched in order by their position.

\begin{pyconcode}
>>> def test(a, b=5, c, d=6):
...     print(a, b, c, d)
  File "<stdin>", line 1
SyntaxError: non-default argument follows default argument
>>> def test(a, c, b=5, d=6):
...     print(a, b, c, d)
...
>>> test(2, 3)
2 5 3 6
>>> test(2, 3, 10)
2 10 3 6
>>> test(2, 3, 10, 100)
2 10 3 100
\end{pyconcode}

We call the arguments that are non-default arguments \emph{positional
arguments}, while the default arguments are called \emph{keyword arguments}.

Sometimes we might want to include some default arguments while excluding
others. We do this by specifying which of the arguments we wish to pass a value
to:

\begin{pyconcode}
>>> def test(a=0, b=0, c=0, d=0):
...     print(a, b, c, d)
...
>>> test(c=10)
0 0 10 0
>>> test(d=5, c=6)
0 0 6 5
\end{pyconcode}

\begin{itemize}
  \item If a function's caller supplies both positional and keyword arguments,
    all positional parameters must precede all keyword parameters.
  \item Keyword parameters may occur in any order.
  \item The function call must supply at least as many parameters as the
    function has positional arguments.
  \item If the caller supplies more positional parameters than the function has
    positional arguments, parameters are matched with keyword arguments
    according to their position.
\end{itemize}

\subsection{Recursion}
\label{subsec:recur}
Recursion is the idea of a function calling itself. This can be useful in a wide
variety of situations, but it can also be easy to misuse. Let us first look at a
good example:

\begin{python3code}
def factorial(n):
  if n <= 1:
    return 1
  else
    return n * factorial(n-1)
\end{python3code}

Notice that this is just another way to repeat a statement, similar to a loop.
In a loop, we have to specify when to stop repeating something, and we have to
do the same thing with recursion; otherwise, it would run forever.

As an example, see this simple \emph{bad} case of recursion:

\begin{python3code}
def test():
    test()
\end{python3code}

If this function is called, it will continue to call itself infinitely until
Python prints out an error message and quits. This is because the function calls
itself every time it is called -- there is no case where the function ends
without calling itself again. The case in which the recursion ends is called a
\emph{base case} and it is important to include one in every recursive function.
Here is a modified version of the above code that allows the initial call to
specify a value limiting how many times the recursive call should be made:

\begin{python3code}
def test(x):
    if x > 0:
        test(x - 1)
\end{python3code}

As an example of a more complex way of using recursion to solve a problem, here
is an example of a recursive function that will sum a list where each element is
either an integer or another list containing integers and lists:

\begin{python3code}
def sum_lists(x):
    total = 0
    for i in x:
        if type(i) == int:
            total += i
        else:
            total += sum_lists(i)
    return total

data = [1, 2, [3, 4, [5, 6], 7], 8, 9]
print(sum_lists(data))
\end{python3code}

This program outputs \pythoninline{45}, which is the sum of all the elements of the
list, including all of the lists inside it. Notice that the base case in this
function is if every element of the list passed to \pythoninline{sum_lists} is an
\pythoninline{int}.

\subsection{Summary}

\begin{description}
\item[Default arguments] Function arguments may have a default value, for
    example:
    \begin{python3code}
import time
def print_error(message, now = time.time()):
  print("Error: {} at time {}!".format(message, now))

print_error('All wrong')
print_error('All more wrong', now = 1.5)
    \end{python3code}

    Default arguments must be specified after arguments that do not have
    defaults.

  \item[Recursion] Recursion occurs when a function calls itself. There must be
    a base case to stop the recursion!
\end{description}

\pagebreak
\section{First-class Functions}
\label{subsec:first}
In Python, functions are so-called ``first-class citizens''. This means
essentially that they can be used in the same way as variables: they can be
assigned to variables, passed to, and returned from functions, as well as stored
in collections. As an example, here is a short program that stores functions
inside of a list, then calls each one of those functions with the same argument.

\begin{python3code}
def print_backwards(string):
    print(string[::-1])

def print_half(string):
    print(string[:len(string)//2])

def print_question(string):
    print("Why do you want me to print '{}'?".format(string))

# Just use the name of the function!
print_functions = [print, print_backwards, print_half, print_question]

for func in print_functions:
    func("Hello, World!")
\end{python3code}

\begin{verbatimcode}
Hello, World!
!dlroW ,olleH
Hello,
Why do you want me to print 'Hello, World!'?
\end{verbatimcode}

This can lead to cool pieces of code where a function takes another function as
an argument and uses it in some way. For example, we can use it to write a
function that composes two other functions and calls them with a given $x$. In
math, this would be like taking a function $f(x)$ and a function $g(x)$ and
composing them to be $h(x) = f(g(x))$. The function below takes $f$ and $g$ and
finds the value of $h$ at a $x$.

\begin{python3code}
def h(f, g, x):
  return f(g(x))
\end{python3code}

\subsection{Map}
\label{subsec:map}
A common operation in Python is to call the same function for every element in a
collection, such as a list or a tuple and then create a new list with the
results. Previously, we would have accomplished this with a \pythoninline{for}
loop:

\begin{python3code}
def double(x):
    return x * 2

data = [5, 23, 76]
new_data = []

for i in data:
    new_data.append(double(i))

print(new_data)
\end{python3code}

\begin{verbatimcode}
[10, 46, 152]
\end{verbatimcode}

If we use the \pythoninline{map} function, we can make this code much more simple:

\begin{python3code}
def double(x):
    return x * 2

data = [5, 23, 76]
data = list(map(double, data))
print(data)
\end{python3code}

\begin{verbatimcode}
[10, 46, 152]
\end{verbatimcode}

The \pythoninline{map} function takes in a function and any iterable data type
(such as a list, tuple, set, etc.) as arguments. It calls that function for
every element of the other value passed, then returns a new data type containing
all of those results. This new data type is a special map object, but in most
cases we are fine with instantly converting it back to our initial data type (in
the above example, a list).

%List comprehensions?

%Reduce has been removed as of Python 3: https://docs.python.org/3.0/whatsnew/3.0.html#builtins
\subsection{Reduce}
\label{subsec:reduce}
\pythoninline{reduce} is a similar function to \pythoninline{map}, but rather than
create a new list with the modified values, \pythoninline{reduce} should be used to
simplify a list into a single value. Similarly to \pythoninline{map},
\pythoninline{reduce} takes in a function and an iterable data type. The difference
is that rather than taking and returning a single value, the function passed to
\pythoninline{reduce} takes in two arguments and returns their combination. As an
example, here is an example of using \pythoninline{reduce} to join a list of
strings into a single larger string.

\begin{python3code}
import functools

def combine(glob, new):
    print("Adding {} to {}.".format(new, glob))
    return glob + new

data = ['this', 'is', 'a', 'test']
data = functools.reduce(combine, data)
print("Final value: {}".format(data))
\end{python3code}

\begin{python3code}
Adding this to is
Adding thisis to a
Adding thisisa to test
Final value: thisisatest
\end{python3code}

Note that we need to run \pythoninline{import functools} in order to have access to
\pythoninline{reduce}.

Let's take a simpler example:
\begin{python3code}
import functools

def add(x, y):
  return x + y

numbers = [1, 3, 4, 6, 7]
s = functools.reduce(add, numbers)
\end{python3code}

\pythoninline!s! is the sum of the values in the list \pythoninline!numbers!. The way
\pythoninline!reduce! works is that it takes the first two values, calls the
\pythoninline!add! function on them, returns the result, and then takes that result
and the next number. In essence, the example above expands to this:
\begin{python3code}
s = add(add(add(add(1, 3), 4), 6), 7)
\end{python3code}

This shows that at each step, you get to see an intermediate value: The first
call to \pythoninline!add! gets the first two values, $1$ and $3$, while the second
call gets the addition of those, $4$ and the next value in the list, $4$. The
third call to \pythoninline!add! gets the result of that, $8$, and the next number,
$6$. This will go on until the list is exhausted.

\subsection{Filter}
\label{subsec:filter}
\pythoninline{filter} is a function that allows easy removal of some elements from
a list. It works similarly to \pythoninline{map}, but the function used needs to
return either \pythoninline{True} or \pythoninline{False}. If it returns
\pythoninline{True} then the given element is included in the final list. If
\pythoninline{False} is returned, then it is not. As an example, here is a short
program that removes all odd elements from a list:

\begin{python3code}
def is_even(x):
    return x % 2 == 0

data = [5, 3, 34, 36, 38, 1, 0, 0, 2]
data = list(filter(is_even, data))
print(data)
\end{python3code}

\begin{verbatimcode}
[34, 36, 38, 0, 0, 2]
\end{verbatimcode}

\subsection{Lambda Functions}
\label{subsec:lambda}
Lambda functions are anonymous functions that are created while the program is
running and are not assigned to a name like normal functions. They can be used
very similarly to normal functions if assigned to a variable:

\begin{pyconcode}
>>> def f(x):
...   return x + 1
... 
>>> g = lambda x: x + 1
>>> f(5)
6
>>> g(5)
6
\end{pyconcode}

In the above case, \pythoninline{f} and \pythoninline{g} are functionally equivalent;
we have just used a different syntax to describe them. 
Both are functions that take in a single value and return that value increased
by one. The primary difference is in syntax: note that the lambda function does
not have a return statement: it will simply return the value computed after the
colon.

Lambda functions are commonly used in conjunction with \pythoninline{map},
\pythoninline{reduce}, and \pythoninline{filter} as shown below:

\begin{python3code}
data = [5, 3, 34, 36, 38, 1, 0, 0, 2]
data = list(filter(lambda x: x % 2 == 0, data))
print(data)
\end{python3code}

\begin{verbatimcode}
[34, 36, 38, 0, 0, 2]
\end{verbatimcode}

This program works just like the example shown in the section on
\pythoninline{filter}, but it does not require us to declare the separate
\pythoninline{is_even} function. Instead, we simply define a lambda inside the call
to \pythoninline{filter}, accomplishing the same thing.

We can similarly use lambda functions to shorten our initial example of
first-class functions:

\begin{python3code}
print_functions = [print,
    lambda string: print(string[::-1]),
    lambda string: print(string[:len(string)//2]),
    lambda string: print("Why do you want me to print '{}'?".format(string))]

for func in print_functions:
    func("Hello, World!")
\end{python3code}

Lambda functions can take more than one argument as well:
\begin{python3code}
multiply = lambda x, y: x * y
print(multiply(3, 5)) # 15
\end{python3code}

One extra use for lambda functions is to modify default sorting behavior. For
example, if we have a list of tuples and try to sort them, Python will sort them
based on their first element, then second if the first elements are the same,
and so on:

\begin{pyconcode}
>>> data = [(5, 2, 4), (6, 3, 2), (4, 4, 4), (3, 3, 3), (5, 3, 10)]
>>> sorted(data)
[(3, 3, 3), (4, 4, 4), (5, 2, 4), (5, 3, 10), (6, 3, 2)]
\end{pyconcode}

However, what if we want to sort these tuples based on their last element? To do
this, we define a special \pythoninline{key} value for \pythoninline{sorted}. Each
value of the tuple will be passed to this function, and then the return values
will be used when sorting in place of the actual value of the tuple:

\begin{pyconcode}
>>> data = [(5, 2, 4), (6, 3, 2), (4, 4, 4), (3, 3, 3), (5, 3, 10)]
>>> sorted(data, key=lambda x: x[-1])
[(6, 3, 2), (3, 3, 3), (5, 2, 4), (4, 4, 4), (5, 3, 10)]
\end{pyconcode}

Notice that \pythoninline!key! is just a default argument to the function 
\pythoninline!sorted!. The default value would be a function that sorts based on
the first element.

Lambda functions also allow us to create functions on the fly. In the
first-class functions section, we composed two functions $f$ and $g$ and found
the value at a point $x$. However, sometimes we do not want the value at $x$, we
want a function that takes in a parameter $x$ and returns the composition. Let's
say, for example, that we want a function $h(x) = sqrt(factorial(x))$. Take
a look at the following code:
\begin{python3code}
import math

def compose(f, g):
  # Creates another function that is the composition of f and g and returns it
  return lambda x: f(g(x))

# Using math.sin and math.cos as first-class citizens
h = compose(math.sqrt, math.factorial)

# compose returns a function, so we can use h like a function!
print(h(6))
print(h(10))

m = compose(math.sin, math.cos)
print(m(3))
print(m(4))
\end{python3code}

\begin{verbatimcode}
26.832815729997478
1904.9409439665053
-0.8360218615377305
-0.6080830096407656
\end{verbatimcode}

If you check the answers, you will find that $\sqrt{ 6! } = \sqrt{ 720 } =
26.832\cdots$ and similarly for the others.

Continuing the code, we could now compose $h$ and \pythoninline!math.log10!
(Logarithm with base 10):
\begin{python3code}
logh = compose(math.log10, h)
print(logh(6)) # 1.4286662482156343
\end{python3code}

We can also plug in a lambda function of our choice to the composition function.
For example, if we wanted a function $p(x) = \sqrt{ x^3 }$:
\begin{python3code}
p = compose(math.sqrt, lambda x: x**3)
print(p(100)) # 1000.0
print(p(10)) # 31.622776601683793
\end{python3code}

\pagebreak

\section{Exercises}
\label{sec:ex}

\begin{warningbox}{Boilerplate}
  Remember that this lab \emph{must} use the
  boilerplate syntax introduced in Lab~5, including the review exercises.
\end{warningbox}

\begin{ex}[palindrome.py] Write a recursive function that determines whether a
    string is a palindrome. A palindrome is a word that is the same forwards and
    backwards, for example ``abba''.
\end{ex}

\begin{ex}[fractions2.py] Using the same tuple representation of fractions as in
    Lab~6, do the following:

    \begin{enumerate}
      \item Write a function \pythoninline!prompt_fractions()! that implements the
        following interface that prompts the user for as many fractions as
        he/she wants to enter:

        \begin{verbatimcode}
Enter fraction: 10/2
Enter fraction: 5/7
Enter fraction: 3/8
Enter fraction: stop
        \end{verbatimcode}

        Use \pythoninline!map! to convert each string, such as ``10/2'' to a
        tuple of integers.

        The function should return a list of tuples. In the example, it would
        return:
        \begin{python3code}
[(10, 2), (5, 7), (3, 8)]
        \end{python3code}

      \item Write a function \pythoninline{min_frac} such that calling
        \pythoninline{functools.reduce(min_frac, list_of_fractions)} will return
        the smallest value fraction in the list.
      \item Write a function \pythoninline{sum_frac} such that calling
        \pythoninline{functools.reduce(sum_frac, list_of_fractions)} will return
        a fraction that is the sum of all the fractions in the list.
      \item Write a function \pythoninline{reduce_frac} that takes in a fraction
        and returns that fraction reduced to its lowest terms. Use this function
        with \pythoninline{map} to reduce a list of fractions.
      \item Write a function \pythoninline{reduced} such that calling
        \pythoninline{filter(reduced, list_of_fractions)} will return a list consisting
        of only fractions that are reduced to their lowest terms.
      \item Write a function \pythoninline!sort_frac! that takes in a list of
        fractions and uses \pythoninline{sorted}
        and a lambda \pythoninline{key} function to sort a list of fractions from
        smallest to largest.
    \end{enumerate}

    Do not use any built-in fraction functions or libraries.

    You should use the following boilerplate code for \texttt{fractions2.py}:
    \begin{python3code}
import functools

# ADD YOUR FUNCTIONS HERE

def main():
  list_fractions = prompt_fractions()

  format_frac = lambda frac : "{}/{}".format(frac[0], frac[1])

  smallest = functools.reduce(min_frac, list_fractions)
  print("Smallest fraction:", format_frac(smallest))

  sumfrac = functools.reduce(sum_frac, list_fractions)
  print("Sum of fractions:", format_frac(sumfrac))

  # reducedfrac = map() ??? You fill in!
  # print("Reduced fractions:", reducedfrac)

  fracsorted = sort_frac(list_fraction)
  print("Sorted fractions:", fracsorted)

if __name__ == '__main__':
  main()
    \end{python3code}
  \end{ex}

  \begin{ex}[rpn.py] Take \texttt{rpn\_calculator.py} from Lab~6 and rewrite
    it to use a dictionary of lambda functions to do the actual operation.
    Operations you should support are $+ - * /$. Also, add a $\sin$ and $\cos$
    operation. For example: $0 \sin$ should give $0$, while $1~2~+~\sin$ should
    give $0.1411200080598672$ (this is $\sin(3)$).
  \end{ex}

%  \item[wrapper.py] Write a function \pythoninline!wrapper! that takes two
%    functions, prints their names, calls the first function, prints its result,
%    calls the second function with the first function's result as an argument,
%    prints its result, and then says ``Done.'' Given a function \pythoninline!f!,
%    you can get its name using \pythoninline!f.__name__!. For example:
%
%    \begin{python3code}
%import math
%print(math.sqrt.__name__) # sqrt
%
%def five():
%  return 5
%
%def square(x):
%  return x**2
%
%wrapper(five, square)
%    \end{python3code}
%
%    \begin{verbatimcode}
%Calling function 'five'
%5
%Calling function 'square'
%25
%Done.
%    \end{verbatimcode}

\pagebreak
\section{Submitting}

You should submit your code as a tarball. It should contain all files
used in the exercises for this lab. The submitted file should be named
\begin{center}
  \texttt{cse107\_firstname\_lastname\_lab8.tar.gz}
\end{center}

\begin{center}
  \textbf{Upload your tarball to Canvas.}
\end{center}

\listoftheorems

\end{document}
