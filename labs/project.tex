% Project: Tanks
%
% CSE/IT 107: Introduction to Programming
% New Mexico Tech
%
% Prepared by Russell White and Christopher Koch
% Spring 2015
\documentclass[11pt]{cselabheader}
\usepackage{caption}

%%%%%%%%%%%%%%%%%% SET TITLES %%%%%%%%%%%%%%%%%%%%%%%%%
\fancyhead[R]{Project: Tanks}
\title{Project: Tanks}

\begin{document}

\maketitle

\hrule

\begin{quotation}
  ``When I see a bird that walks like a duck and swims like a duck and quacks like
  a duck, I call that bird a duck.''
\end{quotation}
\begin{flushright}
  --- James Whitcomb Riley, Duck Enthusiast
\end{flushright}



\begin{quotation}
``Imagination is more important than knowledge.''
\end{quotation}
\begin{flushright}
  --- Albert Einstein
\end{flushright}

\begin{quotation}
``The limits of my language mean the limits of my world.''
\end{quotation}
\begin{flushright}
  --- Ludwig Wittgenstein
\end{flushright}

\hrule


\section{Tanks}

\begin{warningbox}{Dirtbags Tanks}
  This project is inspired by Dirtbags Tanks by Neale Pickett.
  More information is available at \url{http://woozle.org/tanks/intro.html}.
\end{warningbox}

\subsection{Project Overview}
You will be provided with several files comprising the Tank project. This is the
skeleton of a game involving several tanks. It consists of the files
\texttt{game.py}, \texttt{sample\_tank.py}, \texttt{sensor.py},
\texttt{tank.py}, and \texttt{tankutil.py}. For a description of what each of
these files do, stay here. For a description of your assignment, see
Section~\ref{subsec:ex}.

\subsubsection{game.py}
\texttt{game.py} contains two important things: \pythoninline{class Game} and
the \pythoninline{main()} function of the program. \pythoninline{Game}
contains the code that controls the interaction between tanks (collision
detection and firing) as well as the drawing of objects to the
\pythoninlin{tkinter} window. You should not have to worry about how it does
these things.

The \pythoninline{main()} function, however, you will need to edit, slightly.
This is where you will add your own tanks to the \pythoninline{Game} instance
that is already being created. To do this, simply imitate how the
\pythoninline{SampleTank}s are being added. Note that you will also need to add
an import at the start of the file in order to be able to access your tanks from
other files.

\subsubsection{tank.py}
\texttt{tank.py} contains \pythoninlin{class Tank}, which includes all the
methods and attributes needed for general functionality of tanks. All of your
custom tanks should inherit from \pythoninline{Tank}. \emph{You should not edit
this file whatsoever.} Each tank has a large collection of attributes that
control its behavior. Most of these you should not touch, but for some of the
required tanks you may need to change the values of specific attributes in
your custom constructor.

For your tanks, you will need to redefine
\pythoninline{ai(self, delta)}, which is the method called each simulation step
to determine what the tank should do. Inside of this method, the only other
methods you should call are those in the section of \texttt{tank.py} labeled
``\texttt{METHODS TO BE USED BY AI}''. They allow you to set the desired speed
of the tank treads, the desired angle of the turret, and whether the tank should
fire. They also allow you to check the current status of the treads, turret, and
the sensors (described in Section\ref{subsubsec:sensor}). Not using these
methods will result in your tank not behaving properly, as properties such as
tread acceleration, max speed, and turret speed may be ignored.

\subsubsection{tankutil.py}
\texttt{tankutil.py} contains a few miscellaneous functions used by the other
files. It is relatively inconsequential and you won't need to use it, though it
may be useful to look at if you want to set custom colors for your tanks.

\subsubsection{sensor.py}
\label{subsubsec:sensor}
\texttt{sensor.py} contains \pythoninlin{class Sensor}, which is used to define
the sensors for tanks. \pythoninline{Sensor} contains nothing but a connstructor
and a few attributes. The actual logic for sensors resides inside
\texttt{game.py}.

\subsubsection{sample\_tank.py}


\pagebreak
\subsection{Your Mission}
\label{subsec:ex}
You will be submitting 6 tanks that inherit from the \pythoninline{Tank} class
in \texttt{tank.py}. Each of your tanks should override, at the very least,
the \pythoninline{__init__} and \pythoninline{ai(self, delta)} methods of
the parent class.

\begin{ex}[README.txt]
%description of tanks and notes for grader
\end{ex}

\begin{ex}[coward.py]
%runs away from others
\end{ex}

\begin{ex}[charger.py]
%aggressively pursues anything it sees
\end{ex}

\begin{ex}[turret.py]
%can't move, aims at enemies
\end{ex}

\begin{ex}[elephant.py]
%bigger, slower, harder to kill
\end{ex}

\begin{ex}[mouse.py]
%smaller and faster
\end{ex}

\begin{ex}[custom.py]
%your choice
\end{ex}


\subsection{Bug Bounty}


\section{Submitting}
You should submit your code as a tarball. It should contain all files
used in the exercises for this lab. The submitted file should be named
\begin{center}
  \texttt{cse107\_firstname\_lastname\_tanks.tar.gz}
\end{center}

\begin{center}
  \textbf{Upload your tarball to Canvas.}
\end{center}

\listoftheorems

\end{document}
