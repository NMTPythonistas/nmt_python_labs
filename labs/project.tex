% Project: Tanks
%
% CSE/IT 107: Introduction to Programming
% New Mexico Tech
%
% Prepared by Russell White and Christopher Koch
% Spring 2015
\documentclass[11pt]{cselabheader}
\usepackage{caption}

%%%%%%%%%%%%%%%%%% SET TITLES %%%%%%%%%%%%%%%%%%%%%%%%%
\fancyhead[R]{Project: Tanks}
\title{Project: Tanks}

\begin{document}

\maketitle

\hrule

\begin{quotation}
  ``TANK TANK TANK''
\end{quotation}
\begin{flushright}
  --- Intro to \emph{Tank! Tank! Tank!}, Bandai Namco
\end{flushright}

\hrule


\section{Tanks}

\begin{warningbox}{Dirtbags Tanks}
  This project is inspired by Dirtbags Tanks by Neale Pickett.
  More information is available at \url{http://woozle.org/tanks/intro.html}.
\end{warningbox}

You will be required to extend the tank class to three separate
``specialty'' tanks. In every case you will be overriding the
\pythoninline{ai} function to provide new behavior for your tank
type. Additionally, you are able to add sensors to your tank (to
better your AI experience). To understand this, see
\texttt{sample\_tank.py}. In the \pythoninline{__init__} function of
\pythonineline{SampleTank}, two sensors are added to a sensor list. You, as
tankgineer


\subsection{Project Overview}
You will be provided with several files comprising the Tank project. This is the
skeleton of a game involving several tanks. It consists of the files
\texttt{game.py}, \texttt{sample\_tank.py}, \texttt{sensor.py},
\texttt{tank.py}, and \texttt{tankutil.py}. For a description of what each of
these files do, stay here. For a description of your assignment, see
Section~\ref{subsec:ex}.

\subsubsection{game.py}
\texttt{game.py} contains two important things: \pythoninline{class Game} and
the \pythoninline{main()} function of the program. \pythoninline{Game}
contains the code that controls the interaction between tanks (collision
detection and firing) as well as the drawing of objects to the
\pythoninlin{tkinter} window. You should not have to worry about how it does
these things.

The \pythoninline{main()} function, however, you will need to edit, slightly.
This is where you will add your own tanks to the \pythoninline{Game} instance
that is already being created. To do this, simply imitate how the
\pythoninline{SampleTank}s are being added. Note that you will also need to add
an import at the start of the file in order to be able to access your tanks from
other files.

\subsubsection{tank.py}
\texttt{tank.py} contains \pythoninlin{class Tank}, which includes all the
methods and attributes needed for general functionality of tanks. All of your
custom tanks should inherit from \pythoninline{Tank}. \emph{You should not edit
this file whatsoever.} Each tank has a large collection of attributes that
control its behavior. Most of these you should not touch, but for some of the
required tanks you may need to change the values of specific attributes in
your custom constructor.

For your tanks, you will need to redefine
\pythoninline{ai(self, delta)}, which is the method called each simulation step
to determine what the tank should do. Inside of this method, the only other
methods you should call are those in the section of \texttt{tank.py} labeled
``\texttt{METHODS TO BE USED BY AI}''. They allow you to set the desired speed
of the tank treads, the desired angle of the turret, and whether the tank should
fire. They also allow you to check the current status of the treads, turret, and
the sensors (described in Section\ref{subsubsec:sensor}). Not using these
methods will result in your tank not behaving properly, as properties such as
tread acceleration, max speed, and turret speed may be ignored.

\subsubsection{tankutil.py}
\texttt{tankutil.py} contains a few miscellaneous functions used by the other
files. It is relatively inconsequential and you won't need to use it, though it
may be useful to look at if you want to set custom colors for your tanks.

\subsubsection{sensor.py}
\label{subsubsec:sensor}
\texttt{sensor.py} contains \pythoninlin{class Sensor}, which is used to define
the sensors for tanks. \pythoninline{Sensor} contains nothing but a connstructor
and a few attributes. The actual logic for sensors resides inside
\texttt{game.py}.

A sensor is defined from four attributes:
\begin{description}
\item[Direction] A value in degrees that determines what direction the sensor
    is facing. This direction will indicate the middle of the sensor. 0 means
    directly in front of the tank, while 180 means directly behind the tank.
\item[Width] A value in degrees that determines how wide the sensor is. The
    sensor will detect tanks in half this number of degrees in each direction
    from the direction chosen above.
\item[Size] How far the sensor reaches. This distance is in pixels. For
    reference, the default range of a tank's gun is 50.
\item[Tracking] Whether or not the sensor will move as the gun's turret moves.
    This value should be either \pythoninline{True} or \pythoninline{False}. If
    this value is \pythoninline{False}, then the sensor's direction will be
    relative to the tank's current facing. If this value is \pythoninline{True},
    then the sensor's direction will be relative to the turret. This allows you
    to have sensors that, for example, check if there is something to the left
    or right of the turret that could be shot if the turret was moved slightly.
\end{description}

The sensors for your custom tanks should be created inside of the tank's
constructor. To check whether the sensor has detected a tank, use
\pythoninline{self.read_sensor(num)}, where \texttt{num} is the index of the
sensor inside of \pythoninline{self.sensors}.

Each sensor will be shown graphically, and if it detects a tank then it will be
filled in with the color of its parent tank to indicate that it is active.


\subsubsection{sample\_tank.py}
\texttt{sample\_tank.py} is an example of how you might go about making your own
custom tanks. You are free to edit this file, but your final submissions should
be in separate files (as described in Section~\ref{subsec:ex}).
\pythoninline{class SampleTank} defines two methods: \pythoninline{__init__} and
\pythoninline{ai(self, delta)}. \pythoninline{__init__} is be used to
define two sensors: one large one that looks for anything in front of the tank
and one smaller one that follows the turret to check for anything that can be
shot. It also determines whether the turret will turn clockwise or
counter-clockwise.

\pythoninline{ai} is used to control the tank's behavior. The first thing this
tank does is get the current angle of the turret, then set the desired angle to
be higher or lower, depending on what chosen in the constructor. This results
in the turret continuously turning in one direction at a constant rate.

Next, the turret reads from sensor 1 and checks if the turret is able to fire.
Sensor 1 is the smaller sensor that is following the turret. If both values are
true, then the turret fires.

Finally, sensor 0 is checked. This is the large one in front of the tank. If
anything is detected, the tank reverses. If not, it continues forward, with the
left tread at a slightly higher speed than the right one. This results in the
tank slowly turning to the right.

\pagebreak
\subsection{Your Mission}
\label{subsec:ex}
You will be submitting 6 tanks that inherit from the \pythoninline{Tank} class
in \texttt{tank.py}. Each of your tanks should override, at the very least,
the \pythoninline{__init__} and \pythoninline{ai(self, delta)} methods of
the parent class.

\begin{ex}[README.txt]
%description of tanks and notes for grader
\end{ex}

\begin{ex}[coward.py]
%runs away from others
\end{ex}

\begin{ex}[charger.py]
%aggressively pursues anything it sees
\end{ex}

\begin{ex}[turret.py]
%can't move, aims at enemies
\end{ex}

\begin{ex}[elephant.py]
%bigger, slower, harder to kill
\end{ex}

\begin{ex}[mouse.py]
%smaller and faster
\end{ex}

\begin{ex}[custom.py]
%your choice
\end{ex}


\subsection{Bug Bounty}


\section{Submitting}
You should submit your code as a tarball. It should contain all files
used in the exercises for this lab. The submitted file should be named
\begin{center}
  \texttt{cse107\_firstname\_lastname\_tanks.tar.gz}
\end{center}

\begin{center}
  \textbf{Upload your tarball to Canvas.}
\end{center}

\listoftheorems

\end{document}

%%% Local Variables:
%%% mode: latex
%%% TeX-master: t
%%% End:
