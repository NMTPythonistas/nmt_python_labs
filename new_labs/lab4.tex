% LAB 4: Functions
% 
% CSE/IT 107: Introduction to Programming
% New Mexico Tech
% 
% Prepared by Russell White and Christopher Koch
% Fall 2014
\documentclass[11pt]{cselabheader}

%%%%%%%%%%%%%%%%%% SET TITLES %%%%%%%%%%%%%%%%%%%%%%%%%
\fancyhead[R]{Lab 4: Functions}
\title{Lab 4: Functions}

\begin{document}

\maketitle

\hrule
\begin{quotation}
  ``If you don't think carefully, you might believe that programming is just
  typing statements in a programming language.''
\end{quotation}
\begin{flushright}
  --- W. Cunningham
\end{flushright}

\begin{quotation}
  ``Only ugly languages become popular. Python is the exception.''
\end{quotation}
\begin{flushright}
  --- Donald Knuth
\end{flushright}

\hrule

\section{Introduction}

This lab will mainly not contain a lot of text, but instead point you to places
in your text book to read. While we think that repetition is useful, the book
often does a better job of explaining concepts and certainly does a better job
with graphics than we do in the labs. We will call \emph{The Practice of
  Computing Using Python} the \emph{Computing Python} book from now on.

\section{Turtle}

Please see pages 711-719 in \emph{The Practice of Computing Using Python} for
more turtle commands that you will need in this lab; for example, those
involving color.

\begin{warningbox}{Warning}
  The book lists a lot of turtle commands, but does so without prepending the
  method name with \lstinline!turtle.!. This is because instead of using
  \lstinline!import turtle!, they write \lstinline!from turtle import *!. 

  We will not be using this method; please continue to use 
  \lstinline!import turtle! and \lstinline!turtle.method()!. For example, use 
  \lstinline!turtle.begin_fill()! instead of \lstinline!begin_fill()!.
\end{warningbox}

\section{Functions}

Functions are covered by \emph{Chapter 6} in \emph{The Practice of Computing
  Using Python}.

\section{Exercises}
\label{sec:ex}

\begin{description}
  % Computing Python ex 6.9
\item[letters.py] \emph{Computing Python} Chapter 6 Exercise 9

  % Computing Python ex 6.10
\item[fibonacci.py] \emph{Computing Python} Ch 6 Ex 10

  % Computing Python ex 6.12
\item[leap\_year.py] \emph{Computing Python} Ch 6 Ex 12

\item[football.py] \emph{Computing Python} Ch 6 Ex 17

  % Computing Python Project 1
\item[flag.py] \emph{Computing Python} Ch 6 Project 1 (a), (b) \hfill

  Write two functions that draw the United States flag using Turtle.
  These functions must use at least 2 other functions that you write
  to help draw the repetitive parts of the flag.

  One of the functions should draw the flag with the 13 stars arranged
  in rows as shown on page 276 in your book.

  One of the functions should draw the flag with the 13 stars in a circle.
  (Hint: Is it really a circle or is it some other \emph{regular} figure?)
  
  Hint: Take a look at polygons.py and star.py from previous labs.

\item[find.py] \emph{Computing Python} Ch 6 Project 2 (a)

\item[hailstone.py] The Collatz Conjecture states that, for any given $n$,
  you will always reach 1 if you repeat the following steps for long enough:
  \begin{itemize}
  \item If $n$ is odd, multiply it by 3 and add 1.
  \item If $n$ is even, divide it by 2.
  \end{itemize}
  The sequence of values you have along the way is referred to as the hailstone
  sequence for the starting number. For example, the process performed on an initial $n$ of 3 is shown in Table \ref{ex:hailstone}.

  \begin{table}[!ht]
    \centering
    \begin{tabular}{ll}
      \toprule
      $n$ & Rule performed\\
      \midrule
      3 & $3n + 1$\\
      10 & $\frac{n}{2}$\\
      5 & $3n + 1$\\
      16 & $\frac{n}{2}$\\
      8 & $\frac{n}{2}$\\
      4 & $\frac{n}{2}$\\
      2 & $\frac{n}{2}$\\
      1 & Done\\
      \bottomrule
    \end{tabular}
    \caption{Hailstone Sequence for $n = 3$}
    \label{ex:hailstone}
  \end{table}

  The number of steps that must be performed before
  reaching 1 is called the stopping distance. The stopping
  time of 3 in the above example is 7. The stopping
  time of $n = 1$ would be 0.

  Write a function that takes in a starting $n$ for the
  hailstone sequence. The function should then calculate the
  stopping time for that number as well as the largest value
  passed on the way to 1. The above example in Table \ref{ex:hailstone}
  would report a stopping time of 7 and a maximum value of 16.
\end{description} 

\section{Submitting}

Files to submit:
\begin{itemize}
  \item letters.py
  \item fibonacci.py
  \item leap\_year.py
  \item football.py
  \item flag.py
  \item find.py
  \item hailstone.py
\end{itemize}

You may submit your code as either a tarball (instructions below) or as a .zip
file. Either one should contain all files used in the exercises for this lab.
The submitted file should be named either
\texttt{cse107\_firstname\_lastname\_lab4.zip} or
\texttt{cse107\_firstname\_lastname\_lab4.tar.gz} depending on which method you
used.

For Windows, use a tool you like to create a \texttt{.zip} file. The TCC
computers should have \texttt{7z} installed. For Linux, look at lab 1 for
instructions on how to create a tarball or use the ``Archive Manager'' graphical
tool.

\begin{center}
  \textbf{Upload your tarball or .zip file to Canvas.}
\end{center}

\end{document}
