% LAB 4: Functions
% 
% CSE/IT 107: Introduction to Programming
% New Mexico Tech
% 
% Prepared by Russell White and Christopher Koch
% Fall 2014
\documentclass[11pt]{cselabheader}

%%%%%%%%%%%%%%%%%% SET TITLES %%%%%%%%%%%%%%%%%%%%%%%%%
\fancyhead[R]{Lab 4: Functions}
\title{Lab 4: Functions}

\begin{document}

\maketitle

\hrule
\begin{quotation}
``If you don't think carefully, you might believe that programming is just
typing statements in a programming language.''
\end{quotation}
\begin{flushright}
--- W. Cunningham
\end{flushright}

\begin{quotation}
``Only ugly languages become popular. Python is the exception.''
\end{quotation}
\begin{flushright}
--- Donald Knuth
\end{flushright}

\hrule

\section{Introduction}

This lab will mainly not contain a lot of text, but instead point you to places
in your text book to read. While we think that repetition is useful, the book
often does a better job of explaining concepts and certainly does a better job
with graphics than we do in the labs. 

\section{Turtle}

Please see pages 711-719 in \emph{The Practice of Computing Using Python} for
more turtle commands that you will need in this lab; for example, those
involving color.

\begin{warningbox}{Warning}
  The book lists a lot of turtle commands, but does so without prepending the
  method name with \lstinline!turtle.!. This is because instead of using
  \lstinline!import turtle!, they write \lstinline!from turtle import *!. 

  We will not be using this method; please continue to use 
  \lstinline!import turtle! and \lstinline!turtle.method()!. For example, use 
  \lstinline!turtle.begin_fill()! instead of \lstinline!begin_fill()!.
\end{warningbox}

\section{Functions}

Functions are covered by \emph{Chapter 6} in \emph{The Practice of Computing
Using Python}.

\section{Exercises}
\label{sec:ex}

\begin{description}
  \item[functions.py] \hfill
    
    \begin{enumerate}
      % Computing Python ex 6.9
      \item Write a function that takes as input a string and prints the total
        number of vowels and the total number of consonants in the sentence. The
        function returns nothing. Note that the sentence could have special
        characters like dots, dashes, and so on.

      % Computing Python ex 6.12
      \item 

    \end{enumerate}

  \item[hailstone.py]

  % Computing Python Project 1
  \item[flags.py] 

    Write two functions that draw the United States flag using Turtle.
    These functions must use at least 2 other functions that you write
    to help draw the repetitive parts of the flag.

    One of the functions should draw the flag with the 13 stars arranged
    in rows as shown on page 276 in your book.

    One of the functions should draw the flag with the 14 stars in a circle.
    (Hint: Is it really a circle or is it some other \emph{regular} figure?
    
    Hint: Take a look at polygons.py and star.py from previous labs.
\end{description} 

\section{Submitting}

Files to submit:
\begin{itemize}
  \item See all files of Section~\ref{sec:ex}.
\end{itemize}

You may submit your code as either a tarball (instructions below) or as a .zip
file. Either one should contain all files used in the exercises for this lab.
The submitted file should be named either
\texttt{cse107\_firstname\_lastname\_lab4.zip} or
\texttt{cse107\_firstname\_lastname\_lab4.tar.gz} depending on which method you
used.

For Windows, use a tool you like to create a \texttt{.zip} file. The TCC
computers should have \texttt{7z} installed. For Linux, look at lab 1 for
instructions on how to create a tarball or use the ``Archive Manager'' graphical
tool.

\begin{center}
  \textbf{Upload your tarball or .zip file to Canvas.}
\end{center}

\end{document}
