% LAB 9: Sockets
% 
% CSE/IT 107: Introduction to Programming
% New Mexico Tech
% 
% Prepared by Russell White and Christopher Koch
% Fall 2014
\documentclass[11pt]{cselabheader}

%%%%%%%%%%%%%%%%%% SET TITLES %%%%%%%%%%%%%%%%%%%%%%%%%
\fancyhead[R]{Lab 9: Sockets}
\title{Lab 9: Sockets}

\begin{document}

\maketitle

\hrule
\begin{quotation}
``Today, most software exists not to solve a problem, but to interface with other
software.''
\end{quotation}
\begin{flushright}
--- Ian O. Angell
\end{flushright}

\begin{quotation}
``Computer Science is no more about computers than astronomy is about
telescopes.''
\end{quotation}
\begin{flushright}
--- Edsger W. Dijkstra
\end{flushright}

\begin{quotation}
``If people never did silly things, nothing intelligent would ever get done.''
\end{quotation}
\begin{flushright}
--- Ludwig Wittgenstein
\end{flushright}

\hrule

\section{Introduction}

This lab will be two small fun projects involving what computer scientists call
sockets. Really, sockets are network connections: For a computer to be able to
receive a connection, it would open a socket and wait for incoming connections
on that socket (this program would be a server). Another computer (or, perhaps,
the same) may open another socket to connect to the ``remote'' socket (this
program would be the client). Once that has happened, the two machines are
connected. 

Today, you will implement two small programs that connect to a server and solve
a problem. First, you will play a game of Rock Paper Scissor. When you connect
to the server, you will be matched with another program that connected -- this
may take a little while, because you have to wait for someone else to connect.
Once you connect, your program will play Rock Paper Scissors with the other
program. 

In order to communicate, we give you a small ``protocol.'' Generally, a protocol
is just a convention or standard that determines how two programs communicate
with each other. Our protocol will have things like this: If you receive
``wait'' from the server, you have to wait for another player to connect and
just try to keep receiving messages until another one arrives.

\pagebreak
\section{Sockets}
\label{sec:sock}

Sockets are really just simple network connections. To use sockets in Python,
you have to \lstinline!import socket!. We will use the following vocabulary
here: the program accepting network connections from others is called a
\emph{server}, while the program trying to conncet to that is called a
\emph{client}.

In this section, we will only talk about programming a client. The server
program will already be running on a remote computer.

To initiate a network connection, use \lstinline!socket.socket()!:
\begin{lstlisting}
s = socket.socket(socket.AF_INET, socket.SOCK_STREAM)
\end{lstlisting}

Do not mind the arguments to this function. For now, you have to use them and
assume they are black necessary magic.

Just like with files and \lstinline!open!, the variable \lstinline!s! is now
your reference to this network connection. You have to use it every time you are
trying to receive or send something on that network connection.

To then actually connect to a server, use \lstinline!.connect()! and give it a
tuple of hostname or IP \emph{and} port.
\begin{lstlisting}
s.connect( ('arctem.com', 50000) )
\end{lstlisting}
Now this is weird. What are those things? 

Generally, we can say that every computer on the internet has its own IP. An IP
consists of four numbers joined by periods which can go between 0 and 255. For
example, Chris's laptop has the IP \texttt{129.138.6.38} when he is connected to
the WiFi in the computer science department. A lab computer in the last row of 
Speare 23 has the IP \texttt{129.138.47.83}. The computer (``server'') that
hosts Russell's website and some of this lab has the IP \texttt{104.131.56.87}.

A hostname, such as \texttt{arctem.com} (you have seen this in your web
browser!), is just something that is more legible and that translates to an IP.
For example, \texttt{arctem.com} translates to \texttt{104.131.56.87}.
\texttt{paul.kochris.com} translates to \texttt{178.62.37.44}.

Now given this information, you can connect to any computer that is actually
online. But there could be many many programs running on that computer that
accept network connections. If now you connected to that computer, it has to
figure out which program you wanted to connect to. But just the IP does not give
you that information -- the remote computer has no idea.

This is where the port comes in: A server on a remote computer will say ``I
listen to \emph{port} $X$,'' where $X$ is any number between 0 and 65535. Then,
when a client tries to connect to a computer, that machine will look at the port
and ``route'' your connection to the correct program.

For example, when you open a web browser and go to a web page, you are
connecting to the hostname you typed in (for example, \texttt{google.com}) at
port $80$, because that is the standard for HTTP connections. At port $80$ there
is a program that handles the messages for requesting a web page. If you are
using secure browsing, you are connecting to $443$, because that is the standard
for HTTPS connections.

Now that we have connected to a program, we can either receive data or send data
on that connection. 

\begin{lstlisting}
data = s.recv(1024)
\end{lstlisting}

This is how we read data from a socket. \lstinline{s.recv()} is a function that
returns a byte array of the data received at socket \lstinline{s}, with the
argument being the maximum number of bytes received at once. Now, we don't want
a byte array. Byte arrays are weird and scary things (not really, but just
pretend for now -- it's about being in the Halloween spirit!). Instead, we want
a string version of \lstinline{data}. We do this using:

\begin{lstlisting}
data = data.decode()
\end{lstlisting}

This converts \lstinline{data} into a string. It is important, however, that we
don't immediately do this after reading from the socket. First we want to make
sure we actually got something from the socket -- if the other end of the socket
closed and we try to use the data from it, our program will crash!

\begin{lstlisting}
if data:
  data = data.decode()
else:
  print('Remote socket closed.')
\end{lstlisting}

If we want to send our own message, we need to do the reverse of decoding it.
Instead, we need to encode our string into a byte array!

\begin{lstlisting}
msg = "Stuff"
msg = msg.encode()
s.sendall(msg)
\end{lstlisting}

This snippet of code takes our string \lstinline{msg}, encodes it as a byte
array, and then uses \lstinline{s.sendall()} to send it to the socket we're
connected to! Simple!

\begin{lstlisting}[style=python,label={lst:sock},caption={Code that shows usage, but does not
actually make sense}]
import socket
s = socket.socket(socket.AF_INET, socket.SOCK_STREAM)
s.connect( ('arctem.com', 50000) )

data = s.recv(1024)
if data:
  data = data.decode()
else:
  print('Remote socket closed.')
msg = "Stuff"
msg = msg.encode()
s.sendall(msg)
\end{lstlisting}


\begin{warningbox}{Working sockets example}
  To see a working sockets client program, look at
  \texttt{echo\_client.py}. Run it, too! It will, by default, connect to a
  server that will simply send back what you sent it. Send ``echo'', it'll send
  back ``echo''.
\end{warningbox}

\pagebreak
\subsection{Summary}

\begin{table}[!ht]
  \centering
  \begin{tabular}{p{2.5cm} p{3.3cm} p{10cm}}
    \toprule
    \bfseries Function & \bfseries Arguments & \bfseries Purpose \\
    \midrule
    \lstinline!socket.socket! & \lstinline!socket.AF_INET!, \lstinline!socket.SOCK_STREAM! &
    Creates a network connection. Don't alter the arguments. Returns a socket
    reference.\\
    \lstinline!s.socket()! & \lstinline!(host, port)! & Connect to
    \lstinline!host! at \lstinline!port!. \lstinline!s! has to be created by a
    call to \lstinline!socket.socket! (see Listing~\ref{lst:sock})\\
    \lstinline{s.recv()} & \lstinline{size} & Return a byte array of data read from the
    socket with maximum size \lstinline{size}.\\
    \lstinline{s.sendall()} & \lstinline{data} & Send a byte array to the remote socket.\\
    \lstinline!msg.encode()! & & Convert the string \lstinline{msg} into a byte array
    that is compatible with sockets.\\
    \lstinline!data.encode()! & & Convert the byte array \lstinline{data} into a string.\\
    \bottomrule
  \end{tabular}
  \caption{Summary of socket functions \& methods}
  \label{tab:sum}
\end{table}

\pagebreak

\section{Exercises}
\label{sec:ex}

\begin{warningbox}{Boilerplate}
  Remember that this lab \emph{must} use the
  boilerplate syntax introduced in Lab~5.
\end{warningbox}

\begin{warningbox}{Linux}
  Please use Linux in this lab. Below, we will tell you to type in commands to
  the command line -- remember that in Lab 1, you had to type \texttt{idle3} in
  the command line to start Python? Use that same command line to type the
  commands given.
\end{warningbox}

\begin{warningbox}{Server Availability}
  The servers should be available constantly through when the lab is due. If at
  any point you cannot access one of the servers, please email me at
  \href{mailto:rwhite00@nmt.edu}{rwhite00@nmt.edu} and I will fix the problem
  as quickly as possible.
\end{warningbox}

\begin{description}
\item[rps.py] Write a program that connects to the server running at
  \lstinline{arctem.com} port 50001 and plays a game of rock, paper, scissors. The server
  will send the following messages:

  \begin{description}
    \item[name] Next message sent will be your client's display name.
    \item[taken] The name sent is already in use. Repeat sending a name.
    \item[wait] Game has not yet been found (waiting for another player). No
      response required.
    \item[opponent <name>] A game has been found. The opponent's name will be
      inserted for ``<name>''. No response required.
    \item[play] The next message sent should consist solely of ``r'', ``p'', or
      ``s'', depending on whether you wish to play rock, paper, or scissors.
    \item[tie] Your opponent played the same as you, causing a tie. No response
      required.
    \item[win] Your play beat your opponent's, so you won. The next
      message should consist solely of ``y'' or ``n'', indicating your desire to
      play again.
    \item[lose] Your opponent's play beat yours, so you lost. The next message
      should consist solely of ``y'' or ``n'', indicating your desire to play
      again.  
    \item[disconnect] Your opponent disconnected at an unexpected time.  The
      next message should consist solely of ``y'' or ``n'', indicating your
      desire to find a new opponent.
    \item[again] The next message should consist solely of ``y'' or ``n'',
      indicating your desire to play again. This message will only occur if
      invalid input is given for ``win'', ``lose'', or ``disconnect'', so if
      you can be sure that will not occur you do not necessarily need this
      case.
  \end{description}

  Every message sent will be terminated with a ``\textbackslash{}n''. Every message you send
  should be terminated with a ``\textbackslash{}n''. It is possible when you read from a socket
  that you receive multiple commands at once, or less than a full command. Check
  for ``\textbackslash{}n'' to know where each command ends. If you receive part of a command, you
  will have to store that partial command until you get the rest of it.

  The valid messages you can send:

  \begin{description}
  \item[<username>] Your desired username, in response to \lstinline{name} or
    \lstinline{taken}.
  \item[r] A play of ``rock'', in response to \lstinline{play}.
  \item[p] A play of ``paper'', in response to \lstinline{play}.
  \item[s] A play of ``scissors'', in response to \lstinline{play}.
  \item[y] Indication of desire to play again, in response to \lstinline{win},
    \lstinline{lose}, \lstinline{disconnect}, or \lstinline{again}.
  \item[n] Indication of desire to not play again, in response to \lstinline{win},
    \lstinline{lose}, \lstinline{disconnect}, or \lstinline{again}.
  \end{description}

  For each of these server responses, you need to display an appropriate message
  with a prompt for input if appropriate. For example, a ``win'' message might
  output ``Congratulations, you beat <otherplayer>! Do you want to play again?''

\item[maze.py] Write a program that connects to the server running at \lstinline{arctem.com} port 50002
  and automatically navigates a maze generated by the server. The first message you send should be
  your display name (used for the leaderboard). The server will then randomly generate a 10x10
  maze that you must navigate. You will always begin in the top left corner of the maze, facing
  down. You must navigate to the bottom right corner.

  The server will send the following messages:

  \begin{description}
  \item[<left> <forward> <right>] Three space-separated numbers. The first is the number of open
    spaces to your left, the second is the number of open spaces in front of you, and the last is
    the number of open spaces to your right.
  \item[invalid] Sent if the last message you sent was not a valid message or the movement you
    attempted was not possible (such as attempting to move forward into a wall).
  \item[win <steps> <time>] The word ``win'' followed by an integer, then a floating point,
    all space-separated. This indicates you have successfully reached the bottom right corner of
    the maze. After this message is sent, the server will close its end of the socket.
  \end{description}

  The valid messages you can send (after the initial message indicating your name):

  \begin{description}
  \item[forward] Attempt to move one space in the direction you are currently facing.
  \item[backward] Attempt to move one space opposite the direction you are currently facing.
  \item[left] Turn left $90^{\circ}$.
  \item[right] Turn right $90^{\circ}$.
  \end{description}

  Your program's only input should be the desired display name (though you are free to hardcode
  that if you wish). Otherwise, your program should navigate the maze on its own. You are free
  to have any output that you wish to help you get a sense of how your program is doing in
  navigating the maze, though you MUST print out the stats it receives on successful completion of
  the maze (total steps and time taken).

  If you connect to the server on port 50003, it will send you the leaderboard of fastest times
  measured in steps. A simple way to do this in the cmd line is to run
  \lstinline{nc arctem.com 50003}.

  If you connect to the server on port 50004, it will send you the leaderboard of fastest times
  measured in seconds. A simple way to do this in the cmd line is to run
  \lstinline{nc arctem.com 50004}.
  
\end{description}

\pagebreak
\section{Submitting}

Files to submit:
\begin{itemize}
\item rps.py (Section~\ref{sec:ex})
\item maze.py (Section~\ref{sec:ex})
\end{itemize}

You may submit your code as either a tarball (instructions below) or as a .zip
file. Either one should contain all files used in the exercises for this lab.
The submitted file should be named either
\texttt{cse107\_firstname\_lastname\_lab9.zip} or
\texttt{cse107\_firstname\_lastname\_lab9.tar.gz} depending on which method you
used.

For Windows, use a tool you like to create a \texttt{.zip} file. The TCC
computers should have \texttt{7z} installed. For Linux, look at lab 1 for
instructions on how to create a tarball or use the ``Archive Manager'' graphical
tool.

\begin{center}
  \textbf{Upload your tarball or .zip file to Canvas.}
\end{center}

\end{document}
