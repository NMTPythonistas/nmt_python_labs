% LAB 9: Sockets
% 
% CSE/IT 107: Introduction to Programming
% New Mexico Tech
% 
% Prepared by Russell White and Christopher Koch
% Fall 2014
\documentclass[11pt]{cselabheader}
\usepackage{IEEEtrantools}

%%%%%%%%%%%%%%%%%% SET TITLES %%%%%%%%%%%%%%%%%%%%%%%%%
\fancyhead[R]{Lab 9: Sockets}
\title{Lab 9: Sockets}

\begin{document}

\maketitle

\hrule
\begin{quotation}
``Today, most software exists not to solve a problem, but to interface with other
software.''
\end{quotation}
\begin{flushright}
--- Ian O. Angell
\end{flushright}

\begin{quotation}
``Computer Science is no more about computers than astronomy is about
telescopes.''
\end{quotation}
\begin{flushright}
--- Edsger W. Dijkstra
\end{flushright}

\begin{quotation}
``If people never did silly things, nothing intelligent would ever get done.''
\end{quotation}
\begin{flushright}
--- Ludwig Wittgenstein
\end{flushright}

\hrule

\section{Introduction}

\pagebreak
\section{Sockets}
\label{sec:sock}

\pagebreak

\section{Exercises}
\label{sec:ex}

\begin{warningbox}{Boilerplate}
  Remember that this lab \emph{must} use the
  boilerplate syntax introduced in Lab~5.
\end{warningbox}

\begin{description}
\item[rps.py] Write a program that connects to the server running at
  104.131.56.87 port 50000 and plays a game of rock, paper, scissors. The server
  will send the following messages:

  \begin{description}
    \item[username] Next message sent will be your client's display name.
    \item[taken] The name sent is already in use. Repeat sending a name.
    \item[wait] Game has not yet been found (waiting for another player). No
      response required.
    \item[opponent <name>] A game has been found. The opponent's name will be
      inserted for ``<name>''. No response required.
    \item[play] The next message sent should consist solely of ``r'', ``p'', or
      ``s'', depending on whether you wish to play rock, paper, or scissors.
    \item[tie] Your opponent played the same as you, causing a tie. No response
      required.
    \item[win] Your play beat your opponent's, so you won. The next
      message should consist solely of ``y'' or ``n'', indicating your desire to
      play again.
    \item[lose] Your opponent's play beat yours, so you lost. The next message
      should consist solely of ``y'' or ``n'', indicating your desire to play
      again.  
    \item[disconnect] Your opponent disconnected at an unexpected time.  The
      next message should consist solely of ``y'' or ``n'', indicating your
      desire to find a new opponent.
  \end{description}

  For each of these server responses, you need to display an appropriate message
  with a prompt for input if appropriate. For example, a ``win'' message might
  output ``Congratulations, you beat <otherplayer>! Do you want to play again?''
\end{description}

\pagebreak
\section{Submitting}

Files to submit:
\begin{itemize}
  \item rps.py (Section~\ref{sec:ex})
\end{itemize}

You may submit your code as either a tarball (instructions below) or as a .zip
file. Either one should contain all files used in the exercises for this lab.
The submitted file should be named either
\texttt{cse107\_firstname\_lastname\_lab9.zip} or
\texttt{cse107\_firstname\_lastname\_lab9.tar.gz} depending on which method you
used.

For Windows, use a tool you like to create a \texttt{.zip} file. The TCC
computers should have \texttt{7z} installed. For Linux, look at lab 1 for
instructions on how to create a tarball or use the ``Archive Manager'' graphical
tool.

\begin{center}
  \textbf{Upload your tarball or .zip file to Canvas.}
\end{center}

\end{document}
